\section{Проектирование средства оценки планируемой нагрузки на систему ввода-вывода PostgreSQL}

\subsection{Постановка требований к разрабатываемому средству}

Анализ существующих решений показывает, что ни один из подходов не ориентирован напрямую на предварительную оценку I/O-нагрузки без реального выполнения запросов. Поэтому к разрабатываемому средству предъявляются следующие требования:

\begin{itemize}
    \item \textbf{Прогнозирование без запуска нагрузки}: инструмент должен уметь оценивать предполагаемую нагрузку по метаданным базы данных и характеристикам планируемых запросов.
    \item \textbf{Поддержка типовых операций PostgreSQL}: необходимо учитывать особенности выполнения основных операций: вставка, обновление, удаление, выборка.
    \item \textbf{Оценка объема операций чтения и записи}: средство должно отдельно оценивать предполагаемые объемы чтения и записи данных.
    \item \textbf{Модульность и расширяемость}: архитектура решения должна позволять легко адаптировать его к новым версиям PostgreSQL и различным типам приложений.
    \item \textbf{Простота использования}: инструмент должен быть доступен для использования специалистами без глубокого знания внутренней архитектуры PostgreSQL.
\end{itemize}


\subsection{Общая концепция проектируемого решения}
На основании анализа предметной области в Главе 1 было выявлено, что для предварительной оценки I/O-нагрузки требуется разработка средства, которое способно:

\begin{itemize}
    \item Оценивать планируемую нагрузку без фактического выполнения запросов.
    \item Использовать метаданные базы данных PostgreSQL.
    \item Прогнозировать объем операций чтения и записи, исходя из характеристик таблиц, индексов и предполагаемых сценариев работы.
    \item Быть простым в использовании и легко расширяемым.
\end{itemize}

Основная идея разрабатываемого средства заключается в анализе информации о структуре базы данных (размеры таблиц, наличие индексов, схемы запросов) и построении модели, прогнозирующей объем операций ввода-вывода при выполнении определённых типов запросов.

\subsection{Архитектура программного решения}

Разрабатываемое средство оценки планируемой нагрузки на систему ввода-вывода PostgreSQL представляет собой модульный программный комплекс, реализующий механизм анализа и прогнозирования I/O-операций, основанный на структурных характеристиках базы данных и предположениях о поведении запросов. Архитектура решения построена с учетом требований гибкости, расширяемости и совместимости с существующими средствами администрирования PostgreSQL.

\subsubsection{Общие принципы архитектуры}

Архитектура программного средства базируется на следующих принципах:

\begin{itemize}
\item \textbf{Модульность}. Функциональность решения разделена на логически обособленные компоненты: модуль анализа метаданных, модуль моделирования запросов, модуль оценки I/O-нагрузки, модуль визуализации результатов.
\item \textbf{Взаимодействие с PostgreSQL через стандартные интерфейсы}. Все обращения к базе данных осуществляются с использованием официального клиентского API (например, через библиотеку \texttt{psycopg2} для Python), что гарантирует корректность и переносимость решения.
\item \textbf{Конфигурируемость и расширяемость}. Решение допускает задание сценариев моделирования в конфигурационных файлах, а также расширение логики обработки за счёт подключения пользовательских скриптов.
\item \textbf{Прозрачность и воспроизводимость}. Все этапы анализа и оценки фиксируются и могут быть воспроизведены для повторной проверки или модификации параметров.
\end{itemize}

\subsubsection{Структура архитектуры}

% На рисунке \ref{fig\:architecture-diagram} представлена структурная схема архитектуры программного средства.

% \begin{figure}\[h!]
% \centering
% \includegraphics\[width=0.9\textwidth]{architecture\_diagram.png}
% \caption{Структура архитектуры средства оценки I/O-нагрузки}
% \label{fig\:architecture-diagram}
% \end{figure}

Основными компонентами системы являются:

\begin{enumerate}
\item \textbf{Интерфейс конфигурации сценариев моделирования} --- предоставляет пользователю возможность описывать предполагаемые сценарии работы с БД в виде набора параметров: предполагаемые запросы, частота их выполнения, объемы обрабатываемых данных и т. д.


\item \textbf{Модуль сбора метаданных} --- реализует взаимодействие с PostgreSQL для извлечения информации о структуре базы данных: количество строк в таблицах, размер таблиц и индексов, статистики по колонкам и индексам, параметры хранения и фрагментации.

\item \textbf{Модуль анализа и интерпретации запросов} --- осуществляет синтаксический и семантический анализ шаблонов запросов, преобразуя их в представление, пригодное для моделирования объёмов операций чтения и записи.

\item \textbf{Модуль моделирования I/O-нагрузки} --- реализует алгоритмы оценки объема операций чтения/записи для каждого заданного сценария, учитывая структуру таблиц, наличие индексов и статистику. Алгоритмы основаны на эмпирических моделях работы PostgreSQL, таких как правила планирования и оценка затрат.

\item \textbf{Модуль визуализации результатов} --- отображает полученные оценки в виде графиков, таблиц и диаграмм, позволяя пользователю легко интерпретировать результаты и принимать решения по оптимизации структуры базы или характера использования.

\item \textbf{Журналирование и экспорт отчётов} --- ведёт журнал всех проведенных анализов, а также обеспечивает экспорт результатов в форматы CSV, JSON, PDF.


\end{enumerate}

\subsubsection{Взаимодействие компонентов}

Взаимодействие между компонентами системы организовано по принципу потоковой обработки данных:

\begin{enumerate}
\item Пользователь определяет сценарий моделирования через конфигурационный интерфейс.
\item Сценарий передаётся в модуль анализа и интерпретации запросов.
\item Одновременно запускается модуль сбора метаданных, который извлекает текущую информацию о базе данных.
\item Модуль моделирования I/O-нагрузки получает данные от предыдущих модулей и производит расчет ожидаемой нагрузки.
\item Полученные данные передаются в модуль визуализации и экспортируются в требуемом формате.
\end{enumerate}

\subsubsection{Технологический стек}

В качестве основы реализации программного решения выбран язык программирования Python, как один из наиболее подходящих для быстрой разработки и работы с СУБД. Основные используемые технологии и библиотеки:

\begin{itemize}
\item \textbf{PostgreSQL} --- целевая система управления базами данных.
\item \textbf{psycopg2} --- библиотека для взаимодействия с PostgreSQL.
\item \textbf{SQLAlchemy} --- ORM для анализа структуры базы данных.
\item \textbf{pandas, NumPy} --- библиотеки для анализа и обработки числовых данных.
\item \textbf{matplotlib, seaborn} --- визуализация результатов.
\item \textbf{Jinja2} --- генерация отчетов и HTML-документов.
\end{itemize}

\subsubsection{Проектирование базы сценариев моделирования}

Для хранения сценариев и параметров моделирования используется отдельная вспомогательная база или файловая система с форматом JSON/YAML. Каждый сценарий представляет собой структуру, содержащую:

\begin{itemize}
\item Имя сценария и описание.
\item Список запросов (SQL-шаблоны).
\item Параметры исполнения (количество запусков, предполагаемые параметры).
\item Ограничения по времени, объему, таблицам.
\end{itemize}

Такая структура позволяет повторно использовать сценарии и проводить сравнительный анализ между ними.

\subsubsection{Пример сценария моделирования}

Ниже приведен пример описания сценария в формате YAML:

% \begin{minted}\[fontsize=\small]{yaml}
% scenario\_name: daily\_report\_generation
% description: Генерация ежедневных отчетов продаж
% queries:

% * query: "SELECT \* FROM sales WHERE sale\_date = \:date"
%   frequency: 1
%   parameters:
%   date: "2023-10-01"
% * query: "UPDATE inventory SET stock = stock - \:sold WHERE product\_id = \:pid"
%   frequency: 1000
%   parameters:
%   sold: 1
%   pid: 42
%   \end{minted}

\subsubsection{Расширяемость и интеграция}

Система предусматривает возможность интеграции с внешними системами мониторинга и анализа, такими как Prometheus, Zabbix, Grafana. Для этого реализуется экспорт метрик в стандартных форматах (Prometheus exporter, JSON API), а также возможность подключения к REST-интерфейсу.

Дополнительно, архитектура допускает реализацию плагинов на Python, которые могут расширить логику расчета I/O-профиля, включая:

\begin{itemize}
\item Учет работы WAL (журнала транзакций).
\item Учет влияния VACUUM и автозапуска ANALYZE.
\item Влияние параллельных запросов и конкурентного доступа.
\end{itemize}

\subsubsection{Безопасность и отказоустойчивость}

Для обеспечения безопасности:

\begin{itemize}
\item Все соединения с PostgreSQL выполняются с использованием защищенного канала (SSL/TLS).
\item Реализована аутентификация и управление доступом к интерфейсу конфигурации.
\item Ведение логов и проверка целостности входных файлов.
\end{itemize}

Система устойчива к сбоям и сохраняет промежуточные результаты, что позволяет восстанавливать состояние после перезапуска.

\subsubsection{Заключение по архитектуре}

Разработанная архитектура позволяет гибко и масштабируемо решать задачу прогнозирования нагрузки на систему ввода-вывода PostgreSQL. Модульный подход обеспечивает простоту доработки, а использование проверенных технологий делает систему устойчивой и пригодной для промышленной эксплуатации. В следующих разделах рассматривается алгоритмическая реализация модуля оценки нагрузки и приведены примеры его применения.
