\addcontentsline{toc}{section}{\protect\numberline{}ТЕРМИНЫ И ОПРЕДЕЛЕНИЯ}
     
\section*{\centering ТЕРМИНЫ И ОПРЕДЕЛЕНИЯ}

Система управления базами данных (СУБД) -— программное обеспечение, обеспечивающее создание, модификацию, управление и доступ к базам данных. \par

Подсистема ввода-вывода (I/O-система) —- часть вычислительной системы, обеспечивающая обмен данными между оперативной памятью и устройствами хранения данных (дисками, SSD и др.). \par

Нагрузка на систему ввода-вывода —- совокупность операций чтения и записи данных, выполняемых в процессе работы приложения или СУБД. \par

Планируемая нагрузка —- ожидаемое количество операций ввода-вывода, которое должно быть выполнено системой на основании проектных характеристик приложений и базы данных. \par

Производительность СУБД —- способность системы управления базами данных обрабатывать определённый объём транзакций или запросов за единицу времени при заданных ресурсах. \par

Запрос -— команда, направленная к СУБД для получения, изменения или удаления данных. \par

Профилирование нагрузки -— процесс измерения характеристик работы системы для последующего анализа производительности и поиска узких мест. \par

Тестирование производительности -— выполнение заранее определённого набора операций для оценки скорости работы системы при различных условиях нагрузки. \par

Модель нагрузки —- абстрактное представление сценариев взаимодействия пользователя или приложения с базой данных, описывающее частоту и характер операций. \par
