%!TEX TS-program = xelatex
\documentclass[a4paper,14pt]{extarticle}
\usepackage{geometry}
\geometry{
    a4paper,
    top=20mm,
    bottom=20mm,
    left=30mm,    % ГОСТ: левое поле 30 мм
    right=15mm,   % правое поле 15 мм
    bindingoffset=0mm
}

\usepackage{fontspec}
\usepackage{graphicx}
\usepackage{enumitem}
\usepackage{multicol}
\usepackage[hidelinks]{hyperref}
\usepackage{soul} % для выделения текста
\usepackage{minted}
\usepackage{lipsum}
\usepackage{indentfirst}
\usepackage[labelsep=endash]{caption}
\usepackage{titlesec} % Пакет для настройки заголовков
\usepackage{pgfplots}
\usepackage{mwe}
\usepackage{polyglossia} % Для поддержки русского языка
\usepackage{tabularx}
\usepackage{longtable}
\usepackage{booktabs}
\usepackage{float}
\usepackage{csquotes}
\usepackage{pdfpages}

\usepackage{polyglossia}
\setmainlanguage{russian}  % ← Основной язык
\setotherlanguage{english} % ← Дополнительный 
% \usepackage[english, russian]{babel}

% Шрифт TeX Gyre Termes вместо Times New Roman но все ещё по ГОСТ
\usepackage{fontspec}
\setmainfont{PT Serif}[
  Language=Russian,
  Script=Cyrillic
]
% \setmainfont{Times New Roman}


% Межстрочный интервал 1.5 по ГОСТ
\usepackage{setspace}
\onehalfspacing

% Отступ первой строки абзаца 1.25 см
\usepackage{indentfirst}
\setlength{\parindent}{1.25cm}

% Настройка заголовков по ГОСТ
\usepackage{titlesec}
\titleformat{\section}{\normalsize\bfseries\centering}{\thesection}{1em}{}
\titleformat{\subsection}{\normalsize\bfseries\centering}{\thesubsection}{1em}{}
\titlespacing*{\section}{0pt}{24pt}{12pt}  % Между разделами — 24pt до, 12pt после
\titlespacing*{\subsection}{0pt}{18pt}{6pt}

\usepackage[labelsep=endash]{caption}
\captionsetup{
  justification=centering,
  font={bf,small}
}

% Настройка переносов через polyglossia
\PolyglossiaSetup{russian}{
    hyphenmins = {2,3}, % мин. 2 символа до и 3 после переноса
    spelling = modern,
    hyphenation = { % Аналог babelhyphenation
        про-из-во-ди-тель-ность,
        PostgreSQL,
        мас-шта-би-ро-ва-ние
    }
}

\captionsetup{
  labelsep=endash,
  justification=centering,
  font={bf,normal,onehalfspacing,singlespacing,small} % small=12pt в 14pt документе
}

\addto\captionsrussian{\renewcommand{\contentsname}{СОДЕРЖАНИЕ}}
\usepackage{tocloft}
\renewcommand{\cftsecleader}{\cftdotfill{\cftdotsep}}

\usepackage{array} % Для настройки таблиц
\newcolumntype{L}[1]{>{\raggedright\let\newline\\\arraybackslash\hspace{0pt}}p{#1}}
\newcolumntype{C}[1]{>{\centering\let\newline\\\arraybackslash\hspace{0pt}}p{#1}}
\newcolumntype{L}[1]{>{\raggedright\arraybackslash}p{#1}}

\newcommand{\red}[1]{\textcolor{red}{#1}} % для разметки 

\definecolor{markerlightyellow}{RGB}{255,255,200} 
\sethlcolor{markerlightyellow} % установка цвета выделени

\usepackage[
    backend=biber,
    style=gost-numeric,  % Стиль ГОСТ (нумерованный)
    sorting=none,        % Сортировка в порядке упоминания
    language=auto,       % Автоматическое определение языка
    autolang=other,      % Для multilingual библиографии
]{biblatex}


\usepackage{listings}
\usepackage{xcolor} % для цветовой подсветки

% Настройка стиля листинга по ГОСТ
\lstset{
  basicstyle=\ttfamily\normalsize, % ← Размер шрифта ближе к 12 пт
  numbers=left,
  numberstyle=\tiny,
  stepnumber=1,
  numbersep=5pt,
  frame=single,
  breaklines=true,
  tabsize=2
}

\lstdefinelanguage{yaml}{
  keywords={true,false,null,y,n},
  keywordstyle=\color{blue}\bfseries,
  sensitive=false,
  comment=[l]{\#},
  morestring=[b]",
  morestring=[d]',
  stringstyle=\color{red},
  identifierstyle=\color{black},
  moredelim=[l][\color{gray}]{-}
}


\newcounter{listing_cnt}[section]  % Счётчик листингов, привязанный к секциям
\renewcommand{\thelisting}{\thesection.\arabic{listing_cnt}}  % Формат номера: X.Y

% Команда для оформления листинга
\newcommand{\insertlisting}[2]{%
    \refstepcounter{listing_cnt}%
    \begin{center}
        \textbf{Листинг \thelisting} --- #1
    \end{center}
    \lstinputlisting{#2}
}


\addbibresource{references.bib}


\usepackage{appendix}
% Заменяем латинские буквы на кириллические в приложениях
\usepackage{etoolbox}
\usepackage{alphalph}

\begin{document}
    \includepdf[pages={1}]{pdf/title.pdf} % Титульный лист
    \includepdf[width=!, height=!, pages={1-2}]{pdf/abstract.pdf}

    \begin{center}
        \tableofcontents
    \end{center}

    \newpage
    \addcontentsline{toc}{section}{\protect\numberline{}СПИСОК СОКРАЩЕНИЙ И УСЛОВНЫХ ОБОЗНАЧЕНИЙ}
\begin{center}
     \section*{СПИСОК СОКРАЩЕНИЙ И УСЛОВНЫХ ОБОЗНАЧЕНИЙ}
\end{center}

\begin{table}[h]
\centering
\begin{tabular}{|C{3cm}|L{9cm}|} % Вертикальные линии: | между колонками и по краям
\hline
\textbf{Сокращение} & \textbf{Расшифровка} \\ \hline
СУБД       & Система управления базами данных \\ \hline
I/O        & Input/Output (ввод-вывод) \\ \hline
SQL        & Structured Query Language (язык структурированных запросов) \\ \hline
TPS        & Transactions Per Second (транзакций в секунду) \\ \hline
SSD        & Solid-State Drive (накопитель на твёрдотельной памяти) \\ \hline
HDD        & Hard Disk Drive (жёсткий диск) \\ \hline
CPU        & Central Processing Unit (центральный процессор) \\ \hline
RAM        & Random Access Memory (оперативная память) \\ \hline
WAL        & Write-Ahead Logging (журнал предзаписи в PostgreSQL) \\ \hline
DBMS       & Database Management System (англ. эквивалент СУБД) \\ \hline
IOPS       & Input/Output Operations Per Second (операций ввода-вывода в секунду) \\ \hline
RPS        & Requests Per Second (запросов в секунду) \\ \hline
TOAST      & The Oversized-Attribute Storage Technique (техника хранения переразмеренных атрибутов в PostgreSQL) \\ \hline
\end{tabular}
\end{table}

    \newpage
    \addcontentsline{toc}{section}{\protect\numberline{}ТЕРМИНЫ И ОПРЕДЕЛЕНИЯ}
     
\section*{\centering ТЕРМИНЫ И ОПРЕДЕЛЕНИЯ}

Система управления базами данных (СУБД) -— программное обеспечение, обеспечивающее создание, модификацию, управление и доступ к базам данных. \par

Подсистема ввода-вывода (I/O-система) —- часть вычислительной системы, обеспечивающая обмен данными между оперативной памятью и устройствами хранения данных (дисками, SSD и др.). \par

Нагрузка на систему ввода-вывода —- совокупность операций чтения и записи данных, выполняемых в процессе работы приложения или СУБД. \par

Планируемая нагрузка —- ожидаемое количество операций ввода-вывода, которое должно быть выполнено системой на основании проектных характеристик приложений и базы данных. \par

Производительность СУБД —- способность системы управления базами данных обрабатывать определённый объём транзакций или запросов за единицу времени при заданных ресурсах. \par

Запрос -— команда, направленная к СУБД для получения, изменения или удаления данных. \par

Профилирование нагрузки -— процесс измерения характеристик работы системы для последующего анализа производительности и поиска узких мест. \par

Тестирование производительности -— выполнение заранее определённого набора операций для оценки скорости работы системы при различных условиях нагрузки. \par

Модель нагрузки —- абстрактное представление сценариев взаимодействия пользователя или приложения с базой данных, описывающее частоту и характер операций. \par

    \newpage
    \addcontentsline{toc}{section}{\protect\numberline{}ВВЕДЕНИЕ}
\begin{center}
    \section*{\centering ВВЕДЕНИЕ}
\end{center}

В современных ин\-фор\-ма\-ци\-он\-ных си\-сте\-мах ба\-зы дан\-ных за\-ни\-ма\-ют цен\-траль\-ное ме\-сто, 
обе\-спе\-чи\-вая хра\-не\-ние, об\-ра\-бот\-ку и предос\-тав\-ле\-ние ин\-фор\-ма\-ции. При про\-ек\-ти\-ро\-ва\-нии 
и экс\-плуа\-та\-ции си\-стем управ\-ле\-ния ба\-за\-ми дан\-ных (\mbox{СУБД}) од\-ним из кри\-ти\-че\-ски 
важ\-ных фак\-то\-ров яв\-ля\-ет\-ся про\-из\-во\-ди\-тель\-ность. Про\-из\-во\-ди\-тель\-ность, в сво\-ю оче\-редь, 
во мно\-гом опре\-де\-ля\-ет\-ся эф\-фек\-тив\-но\-стью ра\-бо\-ты си\-сте\-мы вво\-да-вы\-во\-да (I/O). \cite{Smelyanskiy2024}

Си\-сте\-ма вво\-да-вы\-во\-да от\-ве\-ча\-ет за вза\-и\-мо\-дей\-ствие меж\-ду опе\-ра\-тив\-ной па\-мя\-тью и дол\-го\-вре\-мен\-ны\-ми 
уст\-рой\-ства\-ми хра\-не\-ния дан\-ных. Нес\-мот\-ря на раз\-ви\-тие тех\-но\-ло\-гий хра\-не\-ния, та\-ких как 
твер\-до\-тель\-ные на\-ко\-пи\-те\-ли (\mbox{SSD}) и си\-сте\-мы хра\-не\-ния в опе\-ра\-тив\-ной па\-мя\-ти 
(\mbox{In-Memory Databases}), про\-бле\-ма ско\-ро\-сти и на\-дёж\-но\-сти опе\-ра\-ций чте\-ния и за\-пи\-си 
ос\-та\-ёт\-ся край\-не ак\-ту\-аль\-ной. \cite{han2024nvme}

Осо\-бен\-но важ\-ной за\-да\-ча оцен\-ки на\-груз\-ки на под\-си\-сте\-му вво\-да-вы\-во\-да ста\-но\-вит\-ся на 
эта\-пах про\-ек\-ти\-ро\-ва\-ния но\-вых си\-стем или мас\-шта\-би\-ро\-ва\-ния су\-ще\-ству\-ющих. Не\-до\-оцен\-ка 
этой на\-груз\-ки мо\-жет при\-ве\-сти к кри\-ти\-че\-ским сбо\-ям в ра\-бо\-те при\-ло\-же\-ний, уве\-ли\-че\-нию 
вре\-ме\-ни от\-кли\-ка, по\-те\-ре дан\-ных и фи\-нан\-со\-вым убыт\-кам. \cite{zhang2021costeffective}  \cite{aws2023reducing} 

На прак\-ти\-ке оцен\-ка пред\-по\-ла\-га\-е\-мой на\-груз\-ки час\-то про\-во\-дит\-ся эм\-пи\-ри\-че\-ски либо на 
ос\-но\-ве опы\-та спе\-ци\-а\-ли\-стов, что не всег\-да поз\-во\-ля\-ет дос\-тичь не\-об\-хо\-ди\-мой точ\-но\-сти. 
Ис\-поль\-зо\-ва\-ние су\-ще\-ству\-ющих ин\-стру\-мен\-тов мо\-ни\-то\-рин\-га, та\-ких как \textit{pgbench}, 
\textit{HammerDB} или си\-сте\-мные сред\-ства ста\-ти\-сти\-ки, так\-же не ре\-ша\-ет про\-бле\-му оцен\-ки 
имен\-но пла\-ни\-ру\-е\-мой, а не фак\-ти\-че\-ской на\-груз\-ки. \cite{vershinin2023optimization, datmt2024tuning}

Современные информационные системы испытывают возрастающие нагрузки, связанные с интенсивным доступом к данным 
и необходимостью оптимальной организации хранения и обработки информации. В среде управления базами данных, 
таких как PostgreSQL, одним из критически важных аспектов производительности является нагрузка на дисковую подсистему, 
возникающая под воздействием различных типов запросов и операций ввода-вывода. 
На рынке труда ИТ-специалистов сейчас особенно высоко ценится умение масштабировать системы, что говорит об актуальности проблемы \cite{Shukhman2022}
Традиционные методы мониторинга и анализа нагрузки опираются на специализированные инструменты и метрики (например, \texttt{iostat}, \texttt{pg\_stat\_io}), 
однако их использование зачастую ограничено невозможностью адекватно предсказывать поведение системы в условиях изменяющихся 
рабочих нагрузок и аномальных ситуаций.


Та\-ким об\-ра\-зом, су\-ще\-ству\-ет яв\-ная по\-треб\-ность в раз\-ра\-бот\-ке средств, ко\-то\-рые поз\-во\-ля\-ли 
бы про\-гно\-зи\-ро\-вать на\-груз\-ку на под\-си\-сте\-му вво\-да-вы\-во\-да Post\-gre\-SQL на ос\-но\-ве ана\-ли\-за 
струк\-ту\-ры ба\-зы дан\-ных, пред\-по\-ла\-га\-е\-мых сце\-на\-ри\-ев её ис\-поль\-зо\-ва\-ния и осо\-бен\-но\-стей 
функ\-ци\-о\-ни\-ро\-ва\-ния са\-мой \mbox{СУБД}.
\vspace{5mm}

\textbf{Цель работы} — разработка средства оценки планируемой нагрузки на систему ввода-вывода СУБД PostgreSQL.\

\vspace{5mm}

\textbf{Для достижения поставленной цели необходимо решить следующие задачи:}
\begin{itemize}[leftmargin=*,align=left]
    \item Про\-анализи\-ровать су\-ще\-ствую\-щие ме\-то\-ды оцен\-ки на\-груз\-ки на си\-сте\-му вво\-да-вы\-во\-да в кон\-тек\-сте ра\-бо\-ты \mbox{СУБД} PostgreSQL.\
    \item Вы\-я\-вить ар\-хи\-тек\-тур\-ные осо\-бен\-но\-сти Post\-gre\-SQL, вли\-яю\-щие на ха\-рак\-тер на\-груз\-ки на под\-си\-сте\-му вво\-да-вы\-во\-да.\
    \item Раз\-ра\-бо\-тать мо\-дель про\-гно\-зи\-ро\-ва\-ния пла\-ни\-ру\-е\-мой I/O-на\-груз\-ки на ос\-но\-ве ана\-ли\-за ме\-та\-дан\-ных и пред\-по\-ла\-га\-е\-мых сце\-на\-ри\-ев ра\-бо\-ты с дан\-ны\-ми.\
    \item Ре\-а\-ли\-зо\-вать про\-грамм\-ное сред\-ство, по\-зво\-ля\-ю\-щее ав\-то\-ма\-ти\-зи\-ро\-вать про\-цесс оцен\-ки.\
    \item Про\-вес\-ти тес\-ти\-ро\-ва\-ние раз\-ра\-бо\-тан\-но\-го сред\-ства на раз\-лич\-ных сце\-на\-ри\-ях ис\-поль\-зо\-ва\-ния ба\-зы дан\-ных.\
\end{itemize}

\vspace{5mm}

\textbf{Объект исследования}: процессы взаимодействия СУБД PostgreSQL с подсистемой ввода-вывода.

\textbf{Предмет исследования}: методы и средства прогнозирования планируемой нагрузки на систему ввода-вывода PostgreSQL.

\vspace{5mm}

\textbf{Актуальность темы} обусловлена необходимостью повышения надёжности проектируемых информационных систем, оптимизации их производительности, а также минимизации затрат на серверное оборудование за счёт более точного планирования ресурсов.

\vspace{5mm}

В ходе работы будут рассмотрены как существующие подходы к мониторингу и оценке нагрузки, так и предложены новые методы прогнозирования на основе анализа метаданных PostgreSQL и особенностей предполагаемой нагрузки.

    \newpage
    \section{Анализ предметной области и существующих решений}

\subsection{Проблематика оценки нагрузки на систему ввода-вывода в СУБД}

В современных информационных системах производительность базы данных является критическим фактором, влияющим на общее качество работы приложений. Одной из ключевых составляющих производительности является взаимодействие СУБД с подсистемой ввода-вывода.

Системы ввода-вывода традиционно считаются одним из узких мест в архитектуре баз данных. В отличие от операций, выполняемых исключительно в оперативной памяти, операции чтения и записи данных на дисковые устройства связаны с гораздо большими задержками. Даже несмотря на распространение быстродействующих накопителей на базе твердотельной памяти (SSD), характер нагрузки на I/O остаётся важнейшим параметром для оценки производительности. \cite{hellerstein2007architecture}

Особую сложность представляет оценка планируемой нагрузки на систему ввода-вывода на ранних этапах проектирования систем. На этой стадии ещё отсутствуют реальные данные эксплуатации, а экспериментальное моделирование может быть слишком затратным по времени и ресурсам.

Таким образом, необходимость в эффективных средствах предварительной оценки нагрузки на подсистему ввода-вывода в системах, использующих СУБД PostgreSQL, представляется очевидной и обоснованной.


\subsection{Способы и средства оценки накрузки на диск} 


\subsubsection{Профилирование реальной нагрузки}
Метод основан на эмпирическом тестировании, когда через генерацию типичных сценариев работы СУБД (с использованием, например, \textit{pgbench} или \textit{HammerDB}) непосредственно измеряется реакция системы на заданную нагрузку. Такой подход позволяет получить конкретные показатели производительности, выявить узкие места и определить, как изменения параметров конфигурации влияют на итоговую производительность базы. Он особенно полезен для оценки критических участков системы и построения сценариев оптимизации, однако его применение обременено необходимостью иметь готовую и соответствующим образом настроенную инфраструктуру, а также требует значительных временных и вычислительных затрат на подготовку тестовых кейсов. При этом данный метод менее применим на начальных этапах разработки, когда еще отсутствует полноразмерное окружение.

\subsubsection{Моделирование нагрузки на основе статистики}
Этот подход базируется на анализе журналов, статистики выполнения запросов и накопленных данных эксплуатации системы. Используя такие инструменты, как \textit{pg\_stat\_statements} и \textit{auto\_explain}, можно провести ретроспективный анализ и выделить закономерности, характерные для конкретных сценариев работы. Подобное статистическое моделирование позволяет оценить интенсивность работы подсистемы ввода-вывода без проведения дополнительных нагрузочных тестов. Однако такой анализ эффективен лишь при наличии достаточного объёма исторических данных, то есть в системах, которые уже работают в производственной среде. В ряде случаев могут возникать сложности в интерпретации статистики, что требует аккуратной настройки инструментов сбора и анализа данных.

\subsubsection{Теоретическое моделирование}
Данный метод опирается на построение аналитических моделей, позволяющих оценивать время выполнения операций ввода-вывода с учётом характеристик таблиц, индексов и структур данных. Примеры таких моделей можно найти в работах Boncz, Stonebraker и других исследователей, где используются математические зависимости для прогнозирования производительности. Преимущество метода заключается в том, что оценка может быть проведена даже без непосредственного проведения нагрузочных тестов, что делает его удобным для ранних стадий проектирования. Вместе с тем требуются глубокие знания внутренней архитектуры СУБД и тщательная калибровка моделей с целью минимизации погрешностей, которые могут возникнуть при упрощении реальных рабочих сценариев.

\subsubsection{Коммерческие решения}
На рынке представлены готовые программные продукты, ориентированные на комплексный мониторинг и анализ производительности баз данных. Такие системы, как \textit{SolarWinds Database Performance Analyzer} или \textit{Pivotal Greenplum Performance Monitor}, обеспечивают не только сбор данных, но и их визуализацию, оповещения в режиме реального времени и рекомендации по оптимизации. Коммерческие решения зачастую интегрированы с более широкими системами управления инфраструктурой, что облегчает их использование в условиях динамически изменяющейся нагрузки. Однако высокий ценовой порог и ориентация на уже работающие системы могут ограничивать применение таких инструментов в проектах, находящихся в стадии разработки или в условиях ограниченного бюджета.

\subsubsection{Исопльзование методов машинного обучения}
можно использовать методы машинного обучения \cite{zaghloul2024correction}
---

Таким образом, представленные подходы обладают своими сильными и слабыми сторонами, а выбор метода оценки I/O-нагрузки должен зависеть от конкретных условий эксплуатации и этапа жизненного цикла СУБД. Эти описания логически дополняют приведённую таблицу, предоставляя более глубокий анализ каждого метода и обосновывая критерии их выбора.


\subsubsection{Сравнительная таблица существующих решений}

Для наглядного сопоставления рассмотрим сводную таблицу, отражающую ключевые характеристики различных подходов:

\begin{table}[H]
\centering
\small
\begin{tabularx}{\textwidth}{@{}p{3.5cm}X>{\raggedright\arraybackslash}p{4.2cm}>{\raggedright\arraybackslash}p{3.5cm}@{}}
\toprule
\textbf{Подход} & \textbf{Типичные инструменты} & \textbf{Преимущества} & \textbf{Ограничения} \\
\midrule
Профилирование реальной нагрузки & \textit{pgbench}, \textit{HammerDB}, кастомные скрипты & Наиболее приближено к реальной работе системы; позволяет наблюдать влияние конкретных операций & Требует развернутой среды; не подходит для ранних стадий проектирования \\
\addlinespace
Моделирование на основе статистики & \textit{pg\_stat\_statements}, \textit{auto\_explain}, \textit{pgBadger} & Позволяет анализировать уже выполненные запросы и выявлять "узкие места" & Не работает без накопленных данных эксплуатации \\
\addlinespace
Теоретическое моделирование & Модели на основе параметров таблиц и индексов (например, Boncz, Stonebraker) & Может использоваться без запуска реальных запросов; полезно на этапе проектирования & Высокая сложность; возможны значительные отклонения от реального поведения \\
\addlinespace
Операционный мониторинг & \textit{iostat}, \textit{vmstat}, \textit{iotop}, \textit{pg\_stat\_io} (PostgreSQL 16+) & Детализированная информация об I/O-активности на уровне ОС и СУБД & Сложно связать данные напрямую с бизнес-логикой или SQL-нагрузкой \\
\addlinespace
Коммерческие решения & \textit{SolarWinds DPA}, \textit{Greenplum Monitor}, \textit{Datadog}, \textit{New Relic} & Готовая визуализация, алерты, прогнозы; часто поддерживают PostgreSQL из коробки & Дорогие; ориентированы на уже работающие системы \\
\bottomrule
\end{tabularx}
\caption{Сравнение подходов к оценке нагрузки на подсистему ввода-вывода в PostgreSQL}
\label{tab:io_approaches}
\end{table}


\subsection{Промежуточные выводы}

В результате проведенного анализа можно сделать следующие выводы:

\begin{itemize}
    \item Эффективная оценка планируемой нагрузки на систему ввода-вывода является важнейшей задачей при проектировании информационных систем на базе PostgreSQL.
    \item Существующие методы в основном ориентированы на постфактум-анализ и не позволяют проводить предварительное прогнозирование нагрузки без фактической генерации запросов.
    \item Разработка специального средства оценки планируемой I/O-нагрузки на основе анализа структуры базы данных и характеристик запросов является актуальной и востребованной задачей.
    \item Для успешной реализации такого средства требуется глубокое понимание архитектуры PostgreSQL, процессов работы с данными и особенностей работы подсистемы ввода-вывода.
\end{itemize}




    \newpage
    \section{Особенности архитектуры PostgreSQL, влияющие на I/O-нагрузку}

PostgreSQL как одна из самых популярных реляционных СУБД имеет архитектурные особенности, которые напрямую влияют на характер нагрузки на подсистему ввода-вывода:

\begin{itemize}
    \item \textbf{Физическая организация хранения данных}: данные в PostgreSQL хранятся в виде таблиц и индексов в файловой системе. Каждая таблица или индекс представлен одним или несколькими файлами.
    \item \textbf{Механизм WAL (Write-Ahead Logging)}: все изменения сначала фиксируются в журнале предзаписи (WAL), что приводит к дополнительной нагрузке на запись.
    \item \textbf{Буферизация данных}: PostgreSQL использует собственный пул буферов для кеширования страниц данных в памяти, что снижает количество прямых обращений к диску.
    \item \textbf{Автоматические процессы обслуживания}: процессы \textit{autovacuum} и \textit{autoanalyze} выполняют фоновую 
                                работу по поддержанию базы данных в оптимальном состоянии, также внося вклад в I/O-нагрузку.
    \item \textbf{Методика выполнения запросов}: особенности планировщика запросов (\textit{Query Planner}) и используемые 
            ыим стратегии сканирования таблиц (sequential scan, index scan) влияют на количество операций чтения данных с диска.
\end{itemize}

Учитывая данные особенности, становится очевидным, что точная оценка I/O-нагрузки должна учитывать не только количество и 
характер запросов, но и внутренние механизмы работы PostgreSQL.

Ниже приведены развернутые описания каждого подхода, представленные выше в таблице. Текст дополняет информацию таблицы, 
раскрывая методологию и особенности каждого метода, не повторяя при этом данные, уже указанные в сравнительной характеристике.


\subsection{Параметры конфигурации влияющие на нагрузку на диск}

\begin{table}[h!]
\centering
\vspace{0.5em}
\begin{tabular}{|L{0.32\textwidth}|L{0.58\textwidth}|}
\hline
\textbf{Параметр} & \textbf{Описание} \\
\hline
shared\_buffers & Размер памяти для кэширования страниц базы данных. Меньший размер увеличивает количество обращений к диску. \\
\hline
work\_mem & Память на операции сортировки и хеширования. При нехватке используется временный файл на диске. \\
\hline
effective\_cache\_size & Оценка доступного файлового кеша ОС для планировщика запросов. \\
\hline
wal\_buffers & Буфер для временного хранения WAL-записей перед записью на диск. \\
\hline
checkpoint\_segments & Количество WAL-сегментов между контрольными точками (устарело, заменено на max\_wal\_size). \\
\hline
max\_wal\_size & Максимальный размер WAL, при достижении которого инициируется чекпоинт. \\
\hline
bgwriter\_lru\_maxpages & Максимальное количество страниц, записываемых background writer-ом за один проход. \\
\hline
bgwriter\_lru\_multiplier & Множитель для вычисления количества страниц, записываемых при нехватке буферов. \\
\hline
synchronous\_commit & Определяет, требуется ли подтверждение записи изменений на диск при коммите транзакции. \\
\hline
temp\_buffers & Память, выделяемая на временные таблицы. При нехватке создаются временные файлы на диске. \\
\hline
log\_temp\_files & Порог логирования временных файлов. Помогает обнаружить чрезмерное использование диска временными файлами. \\
\hline
\end{tabular}
\caption{Параметры PostgreSQL влияющие на нагрузку на диск}
\label{tab:io_params}
\end{table}


\subsection{Структура хранения данных в PostgreSQL}

Система управления базами данных PostgreSQL использует собственную файловую структуру для хранения данных, которая напрямую связана с логической моделью базы данных. 
Понимание физической организации данных является ключевым элементом для администрирования, оптимизации производительности и диагностики проблем, связанных с хранением и доступом к данным.

\subsubsection{Таблицы и файлы: связь объектов БД с файлами}

Каждая база данных в PostgreSQL представлена в файловой системе как каталог в директории \texttt{base/}, в котором хранятся файлы, соответствующие таблицам, индексам и другим объектам. 
Все объекты (таблицы, индексы и т.п.) имеют уникальный OID (object identifier), на основе которого создаётся имя файла. 
Например, таблица с OID 16423 будет храниться в файле \texttt{base/12345/16423}, где 12345 --- OID базы данных.

Каждый такой файл представляет собой физическое представление таблицы. Индексы и TOAST-таблицы также представлены отдельными файлами. \cite{postgres-docs}

\subsubsection{Каталожные таблицы и их физическое представление}

Системные каталоги PostgreSQL (например, \texttt{pg\_class}, \texttt{pg\_attribute}) реализованы как обычные таблицы и хранятся 
в файловой системе наравне с пользовательскими объектами. Они находятся в системной базе данных \texttt{postgres} или \texttt{template1}, 
и также представлены отдельными файлами с числовыми именами (OID).

\subsubsection{Сегментация файлов (segmenting) больших объектов}

В PostgreSQL существует ограничение на максимальный размер одного файла хранения данных --- по умолчанию 1~ГБ. 
Если размер таблицы или индекса превышает 1~ГБ, PostgreSQL автоматически делит его на сегменты. Эти сегменты именуются как 
\texttt{16423.1}, \texttt{16423.2} и т.д., где \texttt{16423} --- базовое имя файла, а суффиксы обозначают номер сегмента.

Это позволяет системе работать с очень большими таблицами, несмотря на ограничения ОС или файловой системы.

\subsubsection{Механизм TOAST и хранение больших данных}

TOAST (The Oversized-Attribute Storage Technique) --- это механизм, предназначенный для хранения очень больших значений полей, 
таких как текст или бинарные объекты. При превышении определённого размера (по умолчанию 2~КБ), такие значения автоматически 
выносятся в отдельную TOAST-таблицу, которая также хранится как обычный файл, связанный с основной таблицей.

TOAST-таблицы используют собственные механизмы сжатия и разбиения данных на чанки (chunks), что позволяет эффективно управлять большими объектами без перегрузки основной таблицы.

\subsubsection{Индексы и их файловая структура}

Индексы в PostgreSQL также реализованы как отдельные файлы. Например, B-Tree индекс хранится в виде страниц фиксированного размера 
(по умолчанию 8~КБ), содержащих ключи и указатели. Как и таблицы, при превышении 1~ГБ, индекс делится на сегменты.

Каждому индексу соответствует запись в каталоге \texttt{pg\_class}, а его физическое имя также строится по OID. 
Структура индекса зависит от его типа: B-Tree, Hash, GiST, GIN и т.д., но в любом случае он представлен в виде одного или нескольких сегментированных файлов.

\subsection{Журналирование и Write-Ahead Logging (WAL) в PostgreSQL}

Одним из основополагающих механизмов обеспечения надежности, согласованности данных и устойчивости к сбоям в системе управления 
базами данных PostgreSQL является организация журналирования на основе протокола Write-Ahead Logging (WAL). 
Этот механизм играет ключевую роль как при обычной работе с данными, так и в процессе восстановления после аварийных ситуаций.

\subsubsection{Принцип Write-Ahead Logging}

Write-Ahead Logging (WAL) --- это протокол, согласно которому любые изменения, вносимые в основные файловые структуры базы данных (таблицы и индексы), 
сначала фиксируются в специальном журнальном файле (WAL-журнале) до того, как сами данные будут непосредственно записаны на диск. 
Это обеспечивает атомарность и долговечность операций, соответствуя принципам управления транзакциями (ACID). \cite{fiskov2025wal}

Ключевой особенностью подхода WAL является следующий момент: ни одно изменение, связанное с данными, 
не считается завершённым, пока информация об этом изменении не будет зафиксирована в журнале. Только после этого модификация 
может быть отражена в основной структуре базы данных. Такой подход позволяет в случае сбоя или неожиданного завершения 
работы системы восстановить базу данных до целостного и согласованного состояния посредством «проигрывания» (replay) 
зафиксированных в WAL операций. \cite{Oparina2024}

\subsubsection{Роль WAL в оценке нагрузки и производительности}

Журналирование посредством WAL напрямую влияет на нагрузку на подсистему ввода-вывода: интенсивность операций по записи в WAL 
часто определяет требования к дисковой подсистеме, особенно при высоких уровнях параллелизма транзакций. 
Анализ объёма, частоты и параметров WAL-записей позволяет прогнозировать и оценивать нагрузку на I/O, 
а также уточнять требования к оборудованию и настройкам сервера. \cite{postgres-docs}

\vspace{1em} Таким образом, механизм Write-Ahead Logging является фундаментом обеспечения надёжности PostgreSQL, а также критически важным фактором для оценки и прогнозирования интегральной нагрузки на систему ввода-вывода при проектировании и эксплуатации СУБД.

\bigskip

Таким образом, физическая организация PostgreSQL напрямую отражает логическую структуру базы данных. 
Глубокое понимание связей между файлами и объектами БД позволяет администратору эффективно управлять хранением, планировать ресурсы и проводить тонкую настройку производительности.


    \newpage
    \section{Проектирование средства оценки планируемой нагрузки на систему ввода-вывода PostgreSQL}

\subsection{Постановка требований к разрабатываемому средству}

Анализ существующих решений показывает, что ни один из подходов не ориентирован напрямую на предварительную оценку I/O-нагрузки без реального выполнения запросов. Поэтому к разрабатываемому средству предъявляются следующие требования:

\begin{itemize}
    \item \textbf{Прогнозирование без запуска нагрузки}: инструмент должен уметь оценивать предполагаемую нагрузку по метаданным базы данных и характеристикам планируемых запросов.
    \item \textbf{Поддержка типовых операций PostgreSQL}: необходимо учитывать особенности выполнения основных операций: вставка, обновление, удаление, выборка.
    \item \textbf{Оценка объема операций чтения и записи}: средство должно отдельно оценивать предполагаемые объемы чтения и записи данных.
    \item \textbf{Модульность и расширяемость}: архитектура решения должна позволять легко адаптировать его к новым версиям PostgreSQL и различным типам приложений.
    \item \textbf{Простота использования}: инструмент должен быть доступен для использования специалистами без глубокого знания внутренней архитектуры PostgreSQL.
\end{itemize}


\subsection{Общая концепция проектируемого решения}
На основании анализа предметной области в Главе 1 было выявлено, что для предварительной оценки I/O-нагрузки требуется разработка средства, которое способно:

\begin{itemize}
    \item Оценивать планируемую нагрузку без фактического выполнения запросов.
    \item Использовать метаданные базы данных PostgreSQL.
    \item Прогнозировать объем операций чтения и записи, исходя из характеристик таблиц, индексов и предполагаемых сценариев работы.
    \item Быть простым в использовании и легко расширяемым.
\end{itemize}

Основная идея разрабатываемого средства заключается в анализе информации о структуре базы данных (размеры таблиц, наличие индексов, схемы запросов) и построении модели, прогнозирующей объем операций ввода-вывода при выполнении определённых типов запросов.

\subsection{Архитектура программного решения}

Разрабатываемое средство оценки планируемой нагрузки на систему ввода-вывода PostgreSQL представляет собой модульный программный комплекс, реализующий механизм анализа и прогнозирования I/O-операций, основанный на структурных характеристиках базы данных и предположениях о поведении запросов. Архитектура решения построена с учетом требований гибкости, расширяемости и совместимости с существующими средствами администрирования PostgreSQL.

\subsubsection{Общие принципы архитектуры}

Архитектура программного средства базируется на следующих принципах:

\begin{itemize}
\item \textbf{Модульность}. Функциональность решения разделена на логически обособленные компоненты: модуль анализа метаданных, модуль моделирования запросов, модуль оценки I/O-нагрузки, модуль визуализации результатов.
\item \textbf{Взаимодействие с PostgreSQL через стандартные интерфейсы}. Все обращения к базе данных осуществляются с использованием официального клиентского API (например, через библиотеку \texttt{psycopg2} для Python), что гарантирует корректность и переносимость решения.
\item \textbf{Конфигурируемость и расширяемость}. Решение допускает задание сценариев моделирования в конфигурационных файлах, а также расширение логики обработки за счёт подключения пользовательских скриптов.
\item \textbf{Прозрачность и воспроизводимость}. Все этапы анализа и оценки фиксируются и могут быть воспроизведены для повторной проверки или модификации параметров.
\end{itemize}

\subsubsection{Структура архитектуры}

% На рисунке \ref{fig\:architecture-diagram} представлена структурная схема архитектуры программного средства.

% \begin{figure}\[h!]
% \centering
% \includegraphics\[width=0.9\textwidth]{architecture\_diagram.png}
% \caption{Структура архитектуры средства оценки I/O-нагрузки}
% \label{fig\:architecture-diagram}
% \end{figure}

Основными компонентами системы являются:

\begin{enumerate}
\item \textbf{Интерфейс конфигурации сценариев моделирования} --- предоставляет пользователю возможность описывать предполагаемые сценарии работы с БД в виде набора параметров: предполагаемые запросы, частота их выполнения, объемы обрабатываемых данных и т. д.


\item \textbf{Модуль сбора метаданных} --- реализует взаимодействие с PostgreSQL для извлечения информации о структуре базы данных: количество строк в таблицах, размер таблиц и индексов, статистики по колонкам и индексам, параметры хранения и фрагментации.

\item \textbf{Модуль анализа и интерпретации запросов} --- осуществляет синтаксический и семантический анализ шаблонов запросов, преобразуя их в представление, пригодное для моделирования объёмов операций чтения и записи.

\item \textbf{Модуль моделирования I/O-нагрузки} --- реализует алгоритмы оценки объема операций чтения/записи для каждого заданного сценария, учитывая структуру таблиц, наличие индексов и статистику. Алгоритмы основаны на эмпирических моделях работы PostgreSQL, таких как правила планирования и оценка затрат.

\item \textbf{Модуль визуализации результатов} --- отображает полученные оценки в виде графиков, таблиц и диаграмм, позволяя пользователю легко интерпретировать результаты и принимать решения по оптимизации структуры базы или характера использования.

\item \textbf{Журналирование и экспорт отчётов} --- ведёт журнал всех проведенных анализов, а также обеспечивает экспорт результатов в форматы CSV, JSON, PDF.


\end{enumerate}

\subsubsection{Взаимодействие компонентов}

Взаимодействие между компонентами системы организовано по принципу потоковой обработки данных:

\begin{enumerate}
\item Пользователь определяет сценарий моделирования через конфигурационный интерфейс.
\item Сценарий передаётся в модуль анализа и интерпретации запросов.
\item Одновременно запускается модуль сбора метаданных, который извлекает текущую информацию о базе данных.
\item Модуль моделирования I/O-нагрузки получает данные от предыдущих модулей и производит расчет ожидаемой нагрузки.
\item Полученные данные передаются в модуль визуализации и экспортируются в требуемом формате.
\end{enumerate}

\subsubsection{Технологический стек}

В качестве основы реализации программного решения выбран язык программирования Python, как один из наиболее подходящих для быстрой разработки и работы с СУБД. Основные используемые технологии и библиотеки:

\begin{itemize}
\item \textbf{PostgreSQL} --- целевая система управления базами данных.
\item \textbf{psycopg2} --- библиотека для взаимодействия с PostgreSQL.
\item \textbf{SQLAlchemy} --- ORM для анализа структуры базы данных.
\item \textbf{pandas, NumPy} --- библиотеки для анализа и обработки числовых данных.
\item \textbf{matplotlib, seaborn} --- визуализация результатов.
\item \textbf{Jinja2} --- генерация отчетов и HTML-документов.
\end{itemize}

\subsubsection{Проектирование базы сценариев моделирования}

Для хранения сценариев и параметров моделирования используется отдельная вспомогательная база или файловая система с форматом JSON/YAML. Каждый сценарий представляет собой структуру, содержащую:

\begin{itemize}
\item Имя сценария и описание.
\item Список запросов (SQL-шаблоны).
\item Параметры исполнения (количество запусков, предполагаемые параметры).
\item Ограничения по времени, объему, таблицам.
\end{itemize}

Такая структура позволяет повторно использовать сценарии и проводить сравнительный анализ между ними.

\subsubsection{Пример сценария моделирования}

Ниже приведен пример описания сценария в формате YAML:

% \begin{minted}\[fontsize=\small]{yaml}
% scenario\_name: daily\_report\_generation
% description: Генерация ежедневных отчетов продаж
% queries:

% * query: "SELECT \* FROM sales WHERE sale\_date = \:date"
%   frequency: 1
%   parameters:
%   date: "2023-10-01"
% * query: "UPDATE inventory SET stock = stock - \:sold WHERE product\_id = \:pid"
%   frequency: 1000
%   parameters:
%   sold: 1
%   pid: 42
%   \end{minted}

\subsubsection{Расширяемость и интеграция}

Система предусматривает возможность интеграции с внешними системами мониторинга и анализа, такими как Prometheus, Zabbix, Grafana. Для этого реализуется экспорт метрик в стандартных форматах (Prometheus exporter, JSON API), а также возможность подключения к REST-интерфейсу.

Дополнительно, архитектура допускает реализацию плагинов на Python, которые могут расширить логику расчета I/O-профиля, включая:

\begin{itemize}
\item Учет работы WAL (журнала транзакций).
\item Учет влияния VACUUM и автозапуска ANALYZE.
\item Влияние параллельных запросов и конкурентного доступа.
\end{itemize}

\subsubsection{Безопасность и отказоустойчивость}

Для обеспечения безопасности:

\begin{itemize}
\item Все соединения с PostgreSQL выполняются с использованием защищенного канала (SSL/TLS).
\item Реализована аутентификация и управление доступом к интерфейсу конфигурации.
\item Ведение логов и проверка целостности входных файлов.
\end{itemize}

Система устойчива к сбоям и сохраняет промежуточные результаты, что позволяет восстанавливать состояние после перезапуска.

\subsubsection{Заключение по архитектуре}

Разработанная архитектура позволяет гибко и масштабируемо решать задачу прогнозирования нагрузки на систему ввода-вывода PostgreSQL. Модульный подход обеспечивает простоту доработки, а использование проверенных технологий делает систему устойчивой и пригодной для промышленной эксплуатации. В следующих разделах рассматривается алгоритмическая реализация модуля оценки нагрузки и приведены примеры его применения.

    \newpage
    \section{Тестирование и апробация разработанного решения}

\subsection{Методика апробации и валидирования разрабатываемого средства}

В целях всесторонней оценки корректности и практической применимости разработанного программного средства 
для оценки планируемой нагрузки на систему ввода-вывода СУБД PostgreSQL была реализована процедура тестирования, 
состоящая из нескольких этапов, включающая автоматизированный сбор статистики ввода-вывода на уровне самой СУБД, 
операционной системы, а также сопоставление полученных результатов с прогнозными значениями, 
рассчитываемыми предлагаемой программой. Для автоматизации получения метрик на выделенном сервере был применён 
специально разработанный скрипт (листинг приведён в приложении), обеспечивающий последовательное выполнение следующих действий:

\begin{itemize} 
    \item выбор тестового сценария из подготовленных примеров; 
    \item автоматизированная подготовка данных в тестовой базе (выполнение DDL-операций и наполнение содержимым); 
    \item запуск серии запросов к PostgreSQL с параметризированной интенсивностью (RPS -- requests per second) и продолжительностью нагрузки; 
    \item сбор статистики ввода-вывода посредством штатных средств \\
        СУБД PostgreSQL (динамические представления \texttt{pg\_stat\_io}, \\
        \texttt{pg\_statio\_user\_tables}), 
        а также средствами операционной системы (\texttt{/proc/PID/io}, \texttt{iostat}); 
    \item накопление и анализ дифференциальных показателей между состояниями до и после нагрузки. 
\end{itemize}

Полученные в процессе автоматических тестовых прогонах метрики сравнивались с результатами, 
выдаваемыми предсказывающим алгоритмом, реализованным в рамках дипломного проекта. Совокупность собранных данных 
позволила осуществить сопоставление точности расчетных и фактических величин нагрузки на I/O-подсистему.

Следует отметить, что в данном разделе изложены только общие принципы и последовательность испытаний. 

Детализированное изложение указанных аспектов позволит объективно оценить эффективность и практическую значимость 
разрабатываемого программного средства в условиях разнообразных сценариев эксплуатации СУБД PostgreSQL.

\subsection{Описание тестовых примеров и сценариев}

\begin{table}[htbp]
    \centering
    \captionsetup{justification=centering}
    \caption{Сравнительная характеристика сценариев тестирования PostgreSQL}
    \label{tab:scenarios_short}
    \begin{tabular}{|l|>{\raggedright}p{4.2cm}|l|l|l|l|}
        \hline
        № & Табличная структура (DDL и объём)    & RPS & Индексы & TOAST & Тип нагрузки \\
        \hline
        1 & users (500\,000), orders (1\,200\,000) & 750 & PK, FK & да & SELECT \\ \hline
        2 & sensor\_data (100\,000\,000) & 250 & PK & нет & SELECT \\ \hline
        3 & accounts (200\,000) & 3\,000 & PK, UNIQUE & нет & SELECT \\ \hline
        4 & logs (растет с 0) & 1\,500 & PK & да & INSERT \\ \hline
    \end{tabular}
    \vspace{0.5em}
\end{table}

\textbf{Пояснения:}
\begin{itemize}
    \item В графе ``TOAST'' указано, требуется ли для сценария механизм хранения крупногабаритных данных (TOAST), см.~\cite{postgres_toast}.
    \item ``PK''~--- первичный ключ; ``FK''~--- внешний ключ; ``UNIQUE''~--- уникальный индекс.
    \item Объёмы приведены в количестве строк таблицы согласно сценариям.
    \item Если поле типа JSONB или TEXT может содержать данные, превышающие одну страницу (8~КБ), PostgreSQL применяет TOAST~\cite{postgres_toast,postgres_docs}.
\end{itemize}

\subsection{Тестовая среда и ресурсы}

Тестирование проводилось на выделенном сервере, предоставленном одним из популярных провайдеров облачных решений. 
Система управления базами данных (СУБД) была развернута на операционной системе Ubuntu без дополнительных оптимизаций конфигурации, 
что позволило оценить её производительность в условиях стандартной настройки. 
Такой подход обеспечил объективность результатов тестирования, исключив влияние сторонних модификаций и специализированных тюнинговых параметров.

\insertlisting{Конфигурация тестового стенда \texttt{stand.yaml}}{code/stand.yaml}

В приведённом ниже фрагменте конфигурационного файла PostgreSQL представлены только те параметры, 
которые были явно заданы в стандартной поставке системы управления базами данных (СУБД). 
Остальные настройки соответствуют значениям по умолчанию, предусмотренным базовой конфигурацией. 
В ходе тестирования параметры СУБД не изменялись, за исключением случаев, когда их модификация была явно указана в условиях эксперимента.

\insertlisting{Часть конфигурации СУБД}{code/stand.yaml}

Более полный файл конфигурации PostgreSQL представлен в (см. \ref{app:postgresql_conf}), если в нем нет искомого параметра, 
то параметр имеет стандартное значение для данной версии СУБД.

\subsubsection{Определение характеристик устройства хранения данных}

Для получения объективных характеристик производительности подсистемы ввода-вывода была проведена 
серия тестов с использованием специализированного инструмента {\tt fio}. Данный инструмент позволяет 
моделировать сценарии работы с диском, приближённые к реальной нагрузке систем управления базами данных. 

Следует отметить, что облачный провайдер не предоставляет информацию о модели и технических характеристиках используемого устройства хранения данных. 
Пользователь получает лишь сведения о лимитах производительности ввода-вывода (IOPS) и типе носителя (SSD), 
что ограничивает возможности для детального анализа физических параметров диска.

В частности, для оценки параметров устройства хранения данных использовалась следующая команда:

\insertlisting{Команда тестирования диска с помощью утилиты fio}{code/fio_test.sh}

Параметры использования ключей {\tt fio} следующие:
\begin{itemize}
    \item \texttt{--randrepeat=1} --- обеспечивается воспроизводимость тестирования за счёт фиксированного генератора случайных чисел;
    \item \texttt{--ioengine=libaio} --- используется асинхронный движок ввода-вывода для максимально полного использования возможностей современной операционной системы;
    \item \texttt{--direct=1} --- операции выполняются в режиме прямого доступа, минуя кэш операционной системы;
    \item \texttt{--gtod\_reduce=1} --- минимизация накладных расходов на получение меток времени;
    \item \texttt{--name=test} --- имя задания;
    \item \texttt{--filename=test} --- путь к тестовому файлу;
    \item \texttt{--bs=4k} --- размер блока ввода-вывода составляет 4 КБ, что соответствует типичному размеру страницы в PostgreSQL;
    \item \texttt{--iodepth=64} --- глубина очереди запросов 64, что имитирует параллельную обработку нескольких запросов;
    \item \texttt{--size=1G} --- размер тестируемого файла составляет 1 ГБ;
    \item \texttt{--readwrite=randrw} --- моделируются смешанные случайные чтения и записи;
    \item \texttt{--rwmixread=75} --- 75\% операций приходится на чтение, остальные 25\% на запись;
    \item \texttt{>} \texttt{disk\_benchmark.txt} --- результаты тестирования выводятся в файл для последующего анализа.
\end{itemize}

Полученные данные были использованы для калибровки и построения модели оценки нагрузки на подсистему ввода-вывода при работе СУБД PostgreSQL.

\insertlisting{Результаты тестирования устройства ввода-вывода}{code/disk_benchmark.txt}

\subsection{Сравнение результатов предсказания с реальными замерами}

\subsection{Анализ точности предсказаний}

\subsection{Анализ ошибок и отклонений}

\subsection{Обсуждение ограничений текущей реализации}

Текущая версия программы прогнозирования нагрузки на диск учитывает только обращения к основным файлам таблицы, 
индексам, TOAST и сопутствующим вспомогательным файлам, основываясь на предполагаемом числе чтений и записей к ним. 
Однако реализация не учитывает целый ряд важных источников нагрузки, таких как операции UPDATE и DELETE, 
а также связанные с ними процессы вакуумирования и autovacuum. Помимо этого, в расчетах не моделируется вклад 
операций логирования (например, дополнительных обращений к WAL), влияние которых может быть заметным при интенсивной 
работе базы данных.

Ещё одним существенным ограничением является то, что модель работает с обобщенными параметрами кэширования 
(cache hit ratio) и не принимает во внимание фактическую конфигурацию буферов, показатели системы хранения 
и механизмы управления памятью PostgreSQL. В реальных условиях существенно влияют дополнительные слои кэширования 
на стороне операционной системы, особенности реализации драйверов и контроллеров, нагрузка от конкурирующих приложений, 
а также алгоритмы распределения ввода-вывода на уровне железа. Всё это может приводить к существенным расхождениям 
между расчетными и фактическими значениями нагрузки.

\subsection{Выводы по результатам апробации}

В ходе апробации программы были выявлены случаи, когда рассчитанные значения нагрузки на дисковую подсистему заметно 
отличались от фактических данных, получаемых в реальных условиях эксплуатации PostgreSQL. Наиболее заметные расхождения 
наблюдаются при интенсивных смешанных нагрузках, а также при активной работе внутренних механизмов СУБД 
(например, autovacuum), которые не учтены в текущей реализации.

Тем не менее, несмотря на эту погрешность, полученные результаты имеют практическую ценность: модель позволяет 
получить ориентировочное представление о возможной нагрузке на диск в зависимости от выбранных сценариев работы с 
данными и уровней кэширования. Использование подобных прогнозов может быть полезным на ранних стадиях проектирования 
системы, при сравнительной оценке разных вариантов размещения данных или при предварительной оценке потенциальных 
проблем с производительностью. В дальнейшем точность прогнозов можно повысить за счёт расширения модели, 
более детального учета всех механизмов работы PostgreSQL и настройки параметров согласно реальным характеристикам оборудования.

    \newpage
    \addcontentsline{toc}{section}{\protect\numberline{}ЗАКЛЮЧЕНИЕ}
\section*{\centering ЗАКЛЮЧЕНИЕ}

В ходе данной работы была спроектирована и реализована программа для прогнозирования нагрузки на диск со стороны 
системы управления базами данных PostgreSQL с целью последующего анализа производительности и обеспечения обоснованного 
выбора аппаратных и программных решений для хранения и работы с большими объемами данных. В процессе выполнения работы 
были проанализированы особенности организации файловой структуры PostgreSQL, изучены принципы функционирования системы 
хранения данных, механизмы обращения к файлам и их влияние на нагрузку ввода-вывода.

Целью разработки являлось создание инструмента, который позволял бы на основании характеристик целевой базы данных 
(размера файлов, предполагаемой активности, паттернов доступа) оценить потенциальную нагрузку на диск, представить её 
в разрезе различных вариантов использования кэша и параметров работы PostgreSQL, а также дать возможность разработчикам 
и администраторам баз данных принимать решения о масштабировании, резервировании ресурсов или оптимизации работы СУБД.

В результате, была реализована программа, формирующая детализированные прогнозы по каждому файлу хранения 
(основные файлы таблиц, индексы, TOAST, вспомогательные файлы), с учётом предполагаемого числа операций чтения и записи, 
сценариев уровня попадания в кэш (cache hit ratio) и оценивающая I/O-нагрузку по таким показателям, как количество 
операций (iops) и пропускная способность (throughput) для различных режимов работы. Программа генерирует структурированный 
и наглядный результат, удобный для последующего анализа и интеграции в документы проектирования или технические отчёты.

\textbf{Практическая ценность работы}
Разработанный инструмент предоставляет удобный способ быстрой оценки дисковой нагрузки при проектировании архитектуры 
информационной системы на базе PostgreSQL. Программа позволяет без необходимости проведения трудоёмких полноценных 
нагрузочных испытаний получить первые ориентировочные цифры: объём ожидаемых обращений к файловой системе, пропускную 
способность, зависимость этих величин от параметров кэширования. Это существенно снижает риски при проектировании, 
упрощает выбор аппаратных решений (тип и объём дисков, организация хранения, резервы под разрастание БД) и формирует 
базу для последующего более точного тестирования и калибровки. Полученные с помощью данного инструмента оценки могут 
использоваться для подготовки технических заданий, согласования бюджета на инфраструктуру, сравнения разных вариантов 
партицирования или индексирования, а также для планирования процедур обслуживания и резервного копирования.

Особенно эффективным представляется применение такой программы на этапах пилотирования проектов, где реальных данных 
еще недостаточно, а сценарии обращения к БД можно только приблизительно оценить. В таких ситуациях верификация 
проектных решений с помощью табличной “прогнозной” модели оказывается не только оправданной, но и позволяет избежать 
выбора заведомо неэффективных стратегий работы с СУБД.

\textbf{Ограничения и риски}
При всей полезности предложенного решения необходимо честно отметить ряд существенных ограничений, которые были 
подробно рассмотрены в соответствующем разделе. Наиболее значимым из них на текущем этапе развития программы является 
упрощение самой модели нагрузки: не учитываются операции UPDATE и DELETE, а также связанные с ними активности 
autovacuum и процессов обслуживания таблиц (reindex, анализ, переупаковка и пр.). Помимо этого, в расчетах полностью 
отсутствует влияние операций логирования, которые в PostgreSQL всегда сопровождают запись в таблицы и могут составлять 
значительную часть I/O-нагрузки.

Еще одной из сложностей является влияние механизмов кэширования как на уровне PostgreSQL, так и со стороны операционной 
системы и аппаратных буферов контроллеров и дисков: реальная нагрузка на устройство может быть существенно ниже или 
выше расчетной в зависимости от особенностей рабочей среды, типа используемой СОСД или гипервизора, физических 
характеристик используемых носителей и др. 
В ходе апробации отмечено, что при ряде тестов расхождение между прогнозной и измеренной нагрузкой может быть весьма 
значительным — это связано как с невозможностью точно угадать реальную структуру обращений к данным, так и с 
поведением внутренних оптимизаторов PostgreSQL, а также непредсказуемым влиянием фоновых процессов.

Вместе с тем автор считает, что предлагаемый подход сохраняет свою ценность, так как получаемые прогнозные данные 
позволяют выстраивать работу с учетом заведомых рисков: если по расчету нагрузка подходит вплотную к предельным 
значениям устройств или инфраструктуры, это сигнал о необходимости закладывать дополнительные резервы, либо 
пересматривать и оптимизировать архитектуру информационной системы до начала эксплуатации и насыщения реальными данными.

\textbf{Основные результаты и перспективы развития}
По итогам работы можно выделить основное достижение — разработан удобный в использовании и расширяемый инструмент 
для прогнозирования дисковой нагрузки, наглядно отображающий вклад каждого компонента хранилища и 
различных сценариев работы. Программа полностью отделена от специфики аппаратной платформы либо специализированных 
настроек и может быть использована как отправная точка в различных проектах, с возможностью адаптации под конкретные 
требования бизнеса и инфраструктуры.

Программа построена на гибкой модели ввода исходных данных и поддерживает расширение спектра анализируемых файлов 
и сценариев работы; результаты выводятся в формате, пригодном для коммуникации между разработчиками, архитекторами 
и обслуживающим персоналом.

Важное направление для дальнейшего развития — реализация более точной модели, учитывающей операции обновления и удаления, 
корректно моделирующей нагрузки от autovacuum и VACUUM FULL, анализирующей влияние WAL (журнала предзаписи) 
и фоновых процессов планирования, а также интеграция результатов реальных нагрузочных тестов для калибровки 
коэффициентов и уточнения входных данных. Следующим логичным шагом могла бы стать интеграция с мониторинговыми 
инструментами, что позволит автоматически сравнивать прогноз и реальные метрики, на лету корректируя модель 
под конкретную производственную систему.

Кроме этого, в перспективе рассматривается возможность автоматизированного формирования рекомендаций по оптимизации 
конфигурации PostgreSQL на основании полученных прогнозов, а также генерация предложений по поэтапному переходу 
на масштабируемые решения (например, sharding, распределенное хранение и др.) при превышении расчетной нагрузки.

\textbf{Заключительное слово}
В заключение стоит подчеркнуть, что поставленная задача создания инструмента для прогноза нагрузки на диск в рамках 
эксплуатации PostgreSQL актуальна как для организаций, разворачивающих новые масштабные ИС с нуля, так и для случаев 
миграции или реорганизации уже существующих проектов. Предложенная модель не подменяет собой полноценных нагрузочных 
тестов и не претендует на абсолютную точность, но позволяет существенно повысить осознанность при проектировании, 
выявить “узкие” места задолго до появления производственных проблем и сэкономить ресурсы на этапе экспериментов и проектирования.

В ходе выполнения данной работы сформирован задел для дальнейших исследований в области автоматизации процессов 
проектирования и сопровождения баз данных, повышения прозрачности и воспроизводимости процессов оценки нагрузки 
и в целом снижения издержек на поддержку и масштабирование корпоративных информационных систем.

В заключение отметим, что как для теоретического, так и для прикладного анализа работы PostgreSQL и её взаимодействия 
с файловой системой важно не только наличие инструментов прогноза, но и практический опыт их применения на реальных 
данных. Проведённая работа подтверждает, что даже простые прогнозные модели позволяют заранее выявить уязвимости 
и тем самым существенно сократить как время на внедрение новых решений, так и затраты на их эксплуатацию.

Таким образом, поставленные в работе задачи можно считать успешно решёнными, а результаты – полезными для последующего 
применения и развития как в рамках исследовательских проектов, так и в промышленной эксплуатации сложных баз данных.

    \newpage
    \printbibliography[title={СПИСОК ИСПОЛЬЗОВАННЫХ ИСТОЧНИКОВ}]
    \newpage
    \begin{appendices}

\renewcommand{\lstlistingname}{Листинг}
\renewcommand{\lstlistlistingname}{Список листингов}

\newcommand{\cyrillicAlph}[1]{%
  \ifcase#1\or А\or Б\or В\or Г\or Д\or Е\or Ж\or З\or И\or К\or Л\or М\or Н\or О\or П\or Р\or С\or Т\or У\or Ф\or Х\or Ц\or Ч\or Ш\or Щ\or Э\or Ю\or Я\else ?\fi}

\makeatletter
\renewcommand{\appendix}{%
  \par
  \setcounter{section}{0}%
  \setcounter{subsection}{0}%
  \renewcommand{\thesection}{\cyrillicAlph{\value{section}}}%
}
\makeatother

\renewcommand{\thesection}{\cyrillicAlph{\value{section}}}

\newcommand{\appsection}[1]{
  \refstepcounter{section}
  \section*{ПРИЛОЖЕНИЕ~\thesection. #1}
  \addcontentsline{toc}{section}{ПРИЛОЖЕНИЕ~\thesection. #1}
}

\appsection{Настройки тестовой среды}
\label{app:postgresql_conf}

\lstinputlisting[
    language=,
    basicstyle=\scriptsize\ttfamily,
    caption={Файл конфигурации \texttt{postgresql.conf}},
    label={lst:postgresql_conf}
]{code/postgresql.conf}

\appsection{Конфигурация для экспериментов}
\subsection{\texttt{example1.yaml}}
\label{app:example1}
\lstinputlisting[
    language=,
    basicstyle=\scriptsize\ttfamily,
    caption={Конфигурация для эксперимента \texttt{example1.yaml}},
    label={lst:example1}
]{code/example1.yaml}

\subsection{\texttt{example2.yaml}}
\label{app:example2}
\lstinputlisting[
    language=,
    basicstyle=\scriptsize\ttfamily,
    caption={Конфигурация для эксперимента \texttt{example2.yaml}},
    label={lst:example2}
]{code/example2.yaml}

\subsection{\texttt{example3.yaml}}
\label{app:example3}
\lstinputlisting[
    language=,
    basicstyle=\scriptsize\ttfamily,
    caption={Конфигурация для эксперимента \texttt{example3.yaml}},
    label={lst:example3}
]{code/example3.yaml}

\subsection{\texttt{example4.yaml}}
\label{app:example4}
\lstinputlisting[
    language=,
    basicstyle=\scriptsize\ttfamily,
    caption={Конфигурация для эксперимента \texttt{example4.yaml}},
    label={lst:example4}
]{code/example4.yaml}

\end{appendices}
    
\end{document}
