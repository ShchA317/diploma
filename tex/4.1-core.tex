\section{Анализ предметной области и существующих решений}

\subsection{Проблематика оценки нагрузки на систему ввода-вывода в СУБД}

В современных информационных системах производительность базы данных является критическим фактором, влияющим на общее качество работы приложений. Одной из ключевых составляющих производительности является взаимодействие СУБД с подсистемой ввода-вывода.

Системы ввода-вывода традиционно считаются одним из узких мест в архитектуре баз данных. В отличие от операций, выполняемых исключительно в оперативной памяти, операции чтения и записи данных на дисковые устройства связаны с гораздо большими задержками. Даже несмотря на распространение быстродействующих накопителей на базе твердотельной памяти (SSD), характер нагрузки на I/O остаётся важнейшим параметром для оценки производительности. \cite{hellerstein2007architecture}

Особую сложность представляет оценка планируемой нагрузки на систему ввода-вывода на ранних этапах проектирования систем. На этой стадии ещё отсутствуют реальные данные эксплуатации, а экспериментальное моделирование может быть слишком затратным по времени и ресурсам.

Таким образом, необходимость в эффективных средствах предварительной оценки нагрузки на подсистему ввода-вывода в системах, использующих СУБД PostgreSQL, представляется очевидной и обоснованной.


\subsection{Способы и средства оценки накрузки на диск} 


\subsubsection{Профилирование реальной нагрузки}
Метод основан на эмпирическом тестировании, когда через генерацию типичных сценариев работы СУБД (с использованием, например, \textit{pgbench} или \textit{HammerDB}) непосредственно измеряется реакция системы на заданную нагрузку. Такой подход позволяет получить конкретные показатели производительности, выявить узкие места и определить, как изменения параметров конфигурации влияют на итоговую производительность базы. Он особенно полезен для оценки критических участков системы и построения сценариев оптимизации, однако его применение обременено необходимостью иметь готовую и соответствующим образом настроенную инфраструктуру, а также требует значительных временных и вычислительных затрат на подготовку тестовых кейсов. При этом данный метод менее применим на начальных этапах разработки, когда еще отсутствует полноразмерное окружение.

\subsubsection{Моделирование нагрузки на основе статистики}
Этот подход базируется на анализе журналов, статистики выполнения запросов и накопленных данных эксплуатации системы. Используя такие инструменты, как \textit{pg\_stat\_statements} и \textit{auto\_explain}, можно провести ретроспективный анализ и выделить закономерности, характерные для конкретных сценариев работы. Подобное статистическое моделирование позволяет оценить интенсивность работы подсистемы ввода-вывода без проведения дополнительных нагрузочных тестов. Однако такой анализ эффективен лишь при наличии достаточного объёма исторических данных, то есть в системах, которые уже работают в производственной среде. В ряде случаев могут возникать сложности в интерпретации статистики, что требует аккуратной настройки инструментов сбора и анализа данных.

\subsubsection{Теоретическое моделирование}
Данный метод опирается на построение аналитических моделей, позволяющих оценивать время выполнения операций ввода-вывода с учётом характеристик таблиц, индексов и структур данных. Примеры таких моделей можно найти в работах Boncz, Stonebraker и других исследователей, где используются математические зависимости для прогнозирования производительности. Преимущество метода заключается в том, что оценка может быть проведена даже без непосредственного проведения нагрузочных тестов, что делает его удобным для ранних стадий проектирования. Вместе с тем требуются глубокие знания внутренней архитектуры СУБД и тщательная калибровка моделей с целью минимизации погрешностей, которые могут возникнуть при упрощении реальных рабочих сценариев.

\subsubsection{Коммерческие решения}
На рынке представлены готовые программные продукты, ориентированные на комплексный мониторинг и анализ производительности баз данных. Такие системы, как \textit{SolarWinds Database Performance Analyzer} или \textit{Pivotal Greenplum Performance Monitor}, обеспечивают не только сбор данных, но и их визуализацию, оповещения в режиме реального времени и рекомендации по оптимизации. Коммерческие решения зачастую интегрированы с более широкими системами управления инфраструктурой, что облегчает их использование в условиях динамически изменяющейся нагрузки. Однако высокий ценовой порог и ориентация на уже работающие системы могут ограничивать применение таких инструментов в проектах, находящихся в стадии разработки или в условиях ограниченного бюджета.

\subsubsection{Исопльзование методов машинного обучения}
можно использовать методы машинного обучения \cite{zaghloul2024correction}
---

Таким образом, представленные подходы обладают своими сильными и слабыми сторонами, а выбор метода оценки I/O-нагрузки должен зависеть от конкретных условий эксплуатации и этапа жизненного цикла СУБД. Эти описания логически дополняют приведённую таблицу, предоставляя более глубокий анализ каждого метода и обосновывая критерии их выбора.


\subsubsection{Сравнительная таблица существующих решений}

Для наглядного сопоставления рассмотрим сводную таблицу, отражающую ключевые характеристики различных подходов:

\begin{table}[H]
\centering
\small
\begin{tabularx}{\textwidth}{@{}p{3.5cm}X>{\raggedright\arraybackslash}p{4.2cm}>{\raggedright\arraybackslash}p{3.5cm}@{}}
\toprule
\textbf{Подход} & \textbf{Типичные инструменты} & \textbf{Преимущества} & \textbf{Ограничения} \\
\midrule
Профилирование реальной нагрузки & \textit{pgbench}, \textit{HammerDB}, кастомные скрипты & Наиболее приближено к реальной работе системы; позволяет наблюдать влияние конкретных операций & Требует развернутой среды; не подходит для ранних стадий проектирования \\
\addlinespace
Моделирование на основе статистики & \textit{pg\_stat\_statements}, \textit{auto\_explain}, \textit{pgBadger} & Позволяет анализировать уже выполненные запросы и выявлять "узкие места" & Не работает без накопленных данных эксплуатации \\
\addlinespace
Теоретическое моделирование & Модели на основе параметров таблиц и индексов (например, Boncz, Stonebraker) & Может использоваться без запуска реальных запросов; полезно на этапе проектирования & Высокая сложность; возможны значительные отклонения от реального поведения \\
\addlinespace
Операционный мониторинг & \textit{iostat}, \textit{vmstat}, \textit{iotop}, \textit{pg\_stat\_io} (PostgreSQL 16+) & Детализированная информация об I/O-активности на уровне ОС и СУБД & Сложно связать данные напрямую с бизнес-логикой или SQL-нагрузкой \\
\addlinespace
Коммерческие решения & \textit{SolarWinds DPA}, \textit{Greenplum Monitor}, \textit{Datadog}, \textit{New Relic} & Готовая визуализация, алерты, прогнозы; часто поддерживают PostgreSQL из коробки & Дорогие; ориентированы на уже работающие системы \\
\bottomrule
\end{tabularx}
\caption{Сравнение подходов к оценке нагрузки на подсистему ввода-вывода в PostgreSQL}
\label{tab:io_approaches}
\end{table}


\subsection{Промежуточные выводы}

В результате проведенного анализа можно сделать следующие выводы:

\begin{itemize}
    \item Эффективная оценка планируемой нагрузки на систему ввода-вывода является важнейшей задачей при проектировании информационных систем на базе PostgreSQL.
    \item Существующие методы в основном ориентированы на постфактум-анализ и не позволяют проводить предварительное прогнозирование нагрузки без фактической генерации запросов.
    \item Разработка специального средства оценки планируемой I/O-нагрузки на основе анализа структуры базы данных и характеристик запросов является актуальной и востребованной задачей.
    \item Для успешной реализации такого средства требуется глубокое понимание архитектуры PostgreSQL, процессов работы с данными и особенностей работы подсистемы ввода-вывода.
\end{itemize}



