\addcontentsline{toc}{section}{\protect\numberline{}ЗАКЛЮЧЕНИЕ}
\section*{\centering ЗАКЛЮЧЕНИЕ}

В работе была спроектирована и реализована программа для прогнозирования дисковой нагрузки со стороны СУБД PostgreSQL. Исследованы особенности организации хранения данных, механизмы обращения к файлам и их влияние на I/O-нагрузку.

Разработка направлена на создание инструмента, позволяющего по характеристикам целевой базы данных оценивать потенциальную нагрузку на диск и формировать прогнозы с учётом различных параметров работы PostgreSQL. 
Полученные результаты позволяют обоснованно подходить к выбору аппаратного обеспечения и архитектурных решений.

Программа генерирует структурированный прогноз по каждому типу хранимых данных (таблицы, индексы, TOAST и др.), учитывая параметры чтения/записи, кэширования, и рассчитывает ключевые показатели IOPS и throughput. 
Итоговые данные пригодны для анализа и интеграции в проектную документацию.

\textbf{Практическая ценность}

Инструмент позволяет оперативно оценивать дисковую нагрузку без проведения трудоёмких нагрузочных испытаний, что снижает проектные риски и облегчает выбор технических решений. 
Прогнозируемые показатели применимы для расчётов объёмов обращения, планирования резервов и обоснования бюджета на инфраструктуру.

Особенно полезным применение инструмента является на этапе пилотирования, когда сценарии доступа к данным известны лишь приблизительно, а точная нагрузка — ещё не определена.

\textbf{Ограничения}

Текущая версия не учитывает операции UPDATE, DELETE и связанные процессы обслуживания (autovacuum, reindex и др.), а также влияние логирования (WAL). 
Кэширование на разных уровнях (СУБД, ОС, контроллеры) также влияет на точность прогнозов. В ряде тестов выявлены расхождения между расчётной и фактической нагрузкой из-за особенностей среды и оптимизаторов PostgreSQL.

Тем не менее, предлагаемый подход сохраняет ценность как инструмент предварительной оценки, позволяющий выявлять потенциальные узкие места и адаптировать архитектуру до внедрения.

\textbf{Результаты и перспективы}

Разработан расширяемый инструмент с гибким вводом параметров и генерацией результатов, пригодных для совместной работы разработчиков, архитекторов и администраторов. Перспективы развития включают:

- учёт операций обновления и удаления;
- моделирование влияния WAL и фоновых процессов;
- интеграцию с системами мониторинга;
- автоматическую калибровку модели по реальным метрикам;
- генерацию рекомендаций по конфигурации и масштабированию.

Таким образом, поставленные цели достигнуты. Разработка предоставляет полезную основу для дальнейших исследований и практического применения в задачах оценки и оптимизации работы PostgreSQL в различных условиях эксплуатации.
