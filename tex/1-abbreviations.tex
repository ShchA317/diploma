\addcontentsline{toc}{section}{\protect\numberline{}СПИСОК СОКРАЩЕНИЙ И УСЛОВНЫХ ОБОЗНАЧЕНИЙ}
\begin{center}
     \section*{СПИСОК СОКРАЩЕНИЙ И УСЛОВНЫХ ОБОЗНАЧЕНИЙ}
\end{center}

\begin{table}[h]
\centering
\begin{tabular}{|C{3cm}|L{9cm}|} % Вертикальные линии: | между колонками и по краям
\hline
\textbf{Сокращение} & \textbf{Расшифровка} \\ \hline
СУБД       & Система управления базами данных \\ \hline
I/O        & Input/Output (ввод-вывод) \\ \hline
SQL        & Structured Query Language (язык структурированных запросов) \\ \hline
TPS        & Transactions Per Second (транзакций в секунду) \\ \hline
SSD        & Solid-State Drive (накопитель на твёрдотельной памяти) \\ \hline
HDD        & Hard Disk Drive (жёсткий диск) \\ \hline
CPU        & Central Processing Unit (центральный процессор) \\ \hline
RAM        & Random Access Memory (оперативная память) \\ \hline
WAL        & Write-Ahead Logging (журнал предзаписи в PostgreSQL) \\ \hline
DBMS       & Database Management System (англ. эквивалент СУБД) \\ \hline
IOPS       & Input/Output Operations Per Second (операций ввода-вывода в секунду) \\ \hline
RPS        & Requests Per Second (запросов в секунду) \\ \hline
TOAST      & The Oversized-Attribute Storage Technique (техника хранения переразмеренных атрибутов в PostgreSQL) \\ \hline
\end{tabular}
\end{table}
