\begin{appendices}

\renewcommand{\lstlistingname}{Листинг}
\renewcommand{\lstlistlistingname}{Список листингов}

\newcommand{\cyrillicAlph}[1]{%
  \ifcase#1\or А\or Б\or В\or Г\or Д\or Е\or Ж\or З\or И\or К\or Л\or М\or Н\or О\or П\or Р\or С\or Т\or У\or Ф\or Х\or Ц\or Ч\or Ш\or Щ\or Э\or Ю\or Я\else ?\fi}

\makeatletter
\renewcommand{\appendix}{%
  \par
  \setcounter{section}{0}%
  \setcounter{subsection}{0}%
  \renewcommand{\thesection}{\cyrillicAlph{\value{section}}}%
}
\makeatother

\renewcommand{\thesection}{\cyrillicAlph{\value{section}}}

\newcommand{\appsection}[1]{
  \refstepcounter{section}
  \section*{ПРИЛОЖЕНИЕ~\thesection. #1}
  \addcontentsline{toc}{section}{ПРИЛОЖЕНИЕ~\thesection. #1}
}

\appsection{Настройки тестовой среды}
\label{app:postgresql_conf}

\lstinputlisting[
    language=,
    basicstyle=\scriptsize\ttfamily,
    caption={Файл конфигурации \texttt{postgresql.conf}},
    label={lst:postgresql_conf}
]{code/postgresql.conf}

\appsection{Конфигурация для экспериментов}
\subsection{\texttt{example1.yaml}}
\label{app:example1}
\lstinputlisting[
    language=,
    basicstyle=\scriptsize\ttfamily,
    caption={Конфигурация для эксперимента \texttt{example1.yaml}},
    label={lst:example1}
]{code/example1.yaml}

\subsection{\texttt{example2.yaml}}
\label{app:example2}
\lstinputlisting[
    language=,
    basicstyle=\scriptsize\ttfamily,
    caption={Конфигурация для эксперимента \texttt{example2.yaml}},
    label={lst:example2}
]{code/example2.yaml}

\subsection{\texttt{example3.yaml}}
\label{app:example3}
\lstinputlisting[
    language=,
    basicstyle=\scriptsize\ttfamily,
    caption={Конфигурация для эксперимента \texttt{example3.yaml}},
    label={lst:example3}
]{code/example3.yaml}

\subsection{\texttt{example4.yaml}}
\label{app:example4}
\lstinputlisting[
    language=,
    basicstyle=\scriptsize\ttfamily,
    caption={Конфигурация для эксперимента \texttt{example4.yaml}},
    label={lst:example4}
]{code/example4.yaml}

\end{appendices}