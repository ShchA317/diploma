\section{Тестирование и апробация разработанного решения}

\subsection{Методика апробации и валидирования разрабатываемого средства}

В целях всесторонней оценки корректности и практической применимости разработанного программного средства 
для оценки планируемой нагрузки на систему ввода-вывода СУБД PostgreSQL была реализована процедура тестирования, 
состоящая из нескольких этапов, включающая автоматизированный сбор статистики ввода-вывода на уровне самой СУБД, 
операционной системы, а также сопоставление полученных результатов с прогнозными значениями, 
рассчитываемыми предлагаемой программой. Для автоматизации получения метрик на выделенном сервере был применён 
специально разработанный скрипт (листинг приведён в приложении), обеспечивающий последовательное выполнение следующих действий:

\begin{itemize} 
    \item выбор тестового сценария из подготовленных примеров; 
    \item автоматизированная подготовка данных в тестовой базе (выполнение DDL-операций и наполнение содержимым); 
    \item запуск серии запросов к PostgreSQL с параметризированной интенсивностью (RPS -- requests per second) и продолжительностью нагрузки; 
    \item сбор статистики ввода-вывода посредством штатных средств \\
        СУБД PostgreSQL (динамические представления \texttt{pg\_stat\_io}, \\
        \texttt{pg\_statio\_user\_tables}), 
        а также средствами операционной системы (\texttt{/proc/PID/io}, \texttt{iostat}); 
    \item накопление и анализ дифференциальных показателей между состояниями до и после нагрузки. 
\end{itemize}

Полученные в процессе автоматических тестовых прогонах метрики сравнивались с результатами, 
выдаваемыми предсказывающим алгоритмом, реализованным в рамках дипломного проекта. Совокупность собранных данных 
позволила осуществить сопоставление точности расчетных и фактических величин нагрузки на I/O-подсистему.

Следует отметить, что в данном разделе изложены только общие принципы и последовательность испытаний. 

Детализированное изложение указанных аспектов позволит объективно оценить эффективность и практическую значимость 
разрабатываемого программного средства в условиях разнообразных сценариев эксплуатации СУБД PostgreSQL.

\subsection{Описание тестовых примеров и сценариев}

\begin{table}[htbp]
    \centering
    \captionsetup{justification=centering}
    \caption{Сравнительная характеристика сценариев тестирования PostgreSQL}
    \label{tab:scenarios_short}
    \begin{tabular}{|l|>{\raggedright}p{4.2cm}|l|l|l|l|}
        \hline
        № & Табличная структура (DDL и объём)    & RPS & Индексы & TOAST & Тип нагрузки \\
        \hline
        1 & users (500\,000), orders (1\,200\,000) & 750 & PK, FK & да & SELECT \\ \hline
        2 & sensor\_data (100\,000\,000) & 250 & PK & нет & SELECT \\ \hline
        3 & accounts (200\,000) & 3\,000 & PK, UNIQUE & нет & SELECT \\ \hline
        4 & logs (растет с 0) & 1\,500 & PK & да & INSERT \\ \hline
    \end{tabular}
    \vspace{0.5em}
\end{table}

\textbf{Пояснения:}
\begin{itemize}
    \item В графе ``TOAST'' указано, требуется ли для сценария механизм хранения крупногабаритных данных (TOAST), см.~\cite{postgres_toast}.
    \item ``PK''~--- первичный ключ; ``FK''~--- внешний ключ; ``UNIQUE''~--- уникальный индекс.
    \item Объёмы приведены в количестве строк таблицы согласно сценариям.
    \item Если поле типа JSONB или TEXT может содержать данные, превышающие одну страницу (8~КБ), PostgreSQL применяет TOAST~\cite{postgres_toast,postgres_docs}.
\end{itemize}

\subsection{Тестовая среда и ресурсы}

Тестирование проводилось на выделенном сервере, предоставленном одним из популярных провайдеров облачных решений. 
Система управления базами данных (СУБД) была развернута на операционной системе Ubuntu без дополнительных оптимизаций конфигурации, 
что позволило оценить её производительность в условиях стандартной настройки. 
Такой подход обеспечил объективность результатов тестирования, исключив влияние сторонних модификаций и специализированных тюнинговых параметров.

\insertlisting{Конфигурация тестового стенда \texttt{stand.yaml}}{code/stand.yaml}

В приведённом ниже фрагменте конфигурационного файла PostgreSQL представлены только те параметры, 
которые были явно заданы в стандартной поставке системы управления базами данных (СУБД). 
Остальные настройки соответствуют значениям по умолчанию, предусмотренным базовой конфигурацией. 
В ходе тестирования параметры СУБД не изменялись, за исключением случаев, когда их модификация была явно указана в условиях эксперимента.

\insertlisting{Часть конфигурации СУБД}{code/stand.yaml}

Более полный файл конфигурации PostgreSQL представлен в (см. \ref{app:postgresql_conf}), если в нем нет искомого параметра, 
то параметр имеет стандартное значение для данной версии СУБД.

\subsubsection{Определение характеристик устройства хранения данных}

Для получения объективных характеристик производительности подсистемы ввода-вывода была проведена 
серия тестов с использованием специализированного инструмента {\tt fio}. Данный инструмент позволяет 
моделировать сценарии работы с диском, приближённые к реальной нагрузке систем управления базами данных. 

Следует отметить, что облачный провайдер не предоставляет информацию о модели и технических характеристиках используемого устройства хранения данных. 
Пользователь получает лишь сведения о лимитах производительности ввода-вывода (IOPS) и типе носителя (SSD), 
что ограничивает возможности для детального анализа физических параметров диска.

В частности, для оценки параметров устройства хранения данных использовалась следующая команда:

\insertlisting{Команда тестирования диска с помощью утилиты fio}{code/fio_test.sh}

Параметры использования ключей {\tt fio} следующие:
\begin{itemize}
    \item \texttt{--randrepeat=1} --- обеспечивается воспроизводимость тестирования за счёт фиксированного генератора случайных чисел;
    \item \texttt{--ioengine=libaio} --- используется асинхронный движок ввода-вывода для максимально полного использования возможностей современной операционной системы;
    \item \texttt{--direct=1} --- операции выполняются в режиме прямого доступа, минуя кэш операционной системы;
    \item \texttt{--gtod\_reduce=1} --- минимизация накладных расходов на получение меток времени;
    \item \texttt{--name=test} --- имя задания;
    \item \texttt{--filename=test} --- путь к тестовому файлу;
    \item \texttt{--bs=4k} --- размер блока ввода-вывода составляет 4 КБ, что соответствует типичному размеру страницы в PostgreSQL;
    \item \texttt{--iodepth=64} --- глубина очереди запросов 64, что имитирует параллельную обработку нескольких запросов;
    \item \texttt{--size=1G} --- размер тестируемого файла составляет 1 ГБ;
    \item \texttt{--readwrite=randrw} --- моделируются смешанные случайные чтения и записи;
    \item \texttt{--rwmixread=75} --- 75\% операций приходится на чтение, остальные 25\% на запись;
    \item \texttt{>} \texttt{disk\_benchmark.txt} --- результаты тестирования выводятся в файл для последующего анализа.
\end{itemize}

Полученные данные были использованы для калибровки и построения модели оценки нагрузки на подсистему ввода-вывода при работе СУБД PostgreSQL.

\insertlisting{Результаты тестирования устройства ввода-вывода}{code/disk_benchmark.txt}

\subsection{Сравнение результатов предсказания с реальными замерами}

\subsection{Анализ точности предсказаний}

\subsection{Анализ ошибок и отклонений}

\subsection{Обсуждение ограничений текущей реализации}

Текущая версия программы прогнозирования нагрузки на диск учитывает только обращения к основным файлам таблицы, 
индексам, TOAST и сопутствующим вспомогательным файлам, основываясь на предполагаемом числе чтений и записей к ним. 
Однако реализация не учитывает целый ряд важных источников нагрузки, таких как операции UPDATE и DELETE, 
а также связанные с ними процессы вакуумирования и autovacuum. Помимо этого, в расчетах не моделируется вклад 
операций логирования (например, дополнительных обращений к WAL), влияние которых может быть заметным при интенсивной 
работе базы данных.

Ещё одним существенным ограничением является то, что модель работает с обобщенными параметрами кэширования 
(cache hit ratio) и не принимает во внимание фактическую конфигурацию буферов, показатели системы хранения 
и механизмы управления памятью PostgreSQL. В реальных условиях существенно влияют дополнительные слои кэширования 
на стороне операционной системы, особенности реализации драйверов и контроллеров, нагрузка от конкурирующих приложений, 
а также алгоритмы распределения ввода-вывода на уровне железа. Всё это может приводить к существенным расхождениям 
между расчетными и фактическими значениями нагрузки.

\subsection{Выводы по результатам апробации}

В ходе апробации программы были выявлены случаи, когда рассчитанные значения нагрузки на дисковую подсистему заметно 
отличались от фактических данных, получаемых в реальных условиях эксплуатации PostgreSQL. Наиболее заметные расхождения 
наблюдаются при интенсивных смешанных нагрузках, а также при активной работе внутренних механизмов СУБД 
(например, autovacuum), которые не учтены в текущей реализации.

Тем не менее, несмотря на эту погрешность, полученные результаты имеют практическую ценность: модель позволяет 
получить ориентировочное представление о возможной нагрузке на диск в зависимости от выбранных сценариев работы с 
данными и уровней кэширования. Использование подобных прогнозов может быть полезным на ранних стадиях проектирования 
системы, при сравнительной оценке разных вариантов размещения данных или при предварительной оценке потенциальных 
проблем с производительностью. В дальнейшем точность прогнозов можно повысить за счёт расширения модели, 
более детального учета всех механизмов работы PostgreSQL и настройки параметров согласно реальным характеристикам оборудования.
