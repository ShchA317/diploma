\section{Тестирование и апробация разработанного решения}

\subsection{Методика апробации и валидирования разрабатываемого средства}

В целях всесторонней оценки корректности и практической применимости разработанного программного средства 
для оценки планируемой нагрузки на систему ввода-вывода СУБД PostgreSQL была реализована процедура тестирования, 
состоящая из нескольких этапов, включающая автоматизированный сбор статистики ввода-вывода на уровне самой СУБД, 
операционной системы, а также сопоставление полученных результатов с прогнозными значениями, 
рассчитываемыми предлагаемой программой. Для автоматизации получения метрик на выделенном сервере был применён 
специально разработанный скрипт (листинг приведён в приложении), обеспечивающий последовательное выполнение следующих действий:

\begin{itemize} 
    \item выбор тестового сценария из подготовленных примеров; 
    \item автоматизированная подготовка данных в тестовой базе (выполнение DDL-операций и наполнение содержимым); 
    \item запуск серии запросов к PostgreSQL с параметризированной интенсивностью (RPS -- requests per second) и продолжительностью нагрузки; 
    \item сбор статистики ввода-вывода посредством штатных средств \\
        СУБД PostgreSQL (динамические представления \texttt{pg\_stat\_io}, \\
        \texttt{pg\_statio\_user\_tables}), 
        а также средствами операционной системы (\texttt{/proc/PID/io}, \texttt{iostat}); 
    \item накопление и анализ дифференциальных показателей между состояниями до и после нагрузки. 
\end{itemize}

Полученные в процессе автоматических тестовых прогонах метрики сравнивались с результатами, 
выдаваемыми предсказывающим алгоритмом, реализованным в рамках дипломного проекта. Совокупность собранных данных 
позволила осуществить сопоставление точности расчетных и фактических величин нагрузки на I/O-подсистему.

Следует отметить, что в данном разделе изложены только общие принципы и последовательность испытаний. 

Детализированное изложение указанных аспектов позволит объективно оценить эффективность и практическую значимость 
разрабатываемого программного средства в условиях разнообразных сценариев эксплуатации СУБД PostgreSQL.

\subsection{Описание тестовых примеров и сценариев}

\begin{table}[htbp]
    \centering
    \captionsetup{justification=centering}
    \caption{Сравнительная характеристика сценариев тестирования PostgreSQL}
    \label{tab:scenarios_short}
    \begin{tabular}{|l|>{\raggedright}p{4.2cm}|l|l|l|l|}
        \hline
        № & Табличная структура (DDL и объём)    & RPS & Индексы & TOAST & Тип нагрузки \\
        \hline
        1 & users (500\,000), orders (1\,200\,000) & 750 & PK, FK & да & SELECT \\ \hline
        2 & sensor\_data (100\,000\,000) & 250 & PK & нет & SELECT \\ \hline
        3 & accounts (200\,000) & 3\,000 & PK, UNIQUE & нет & SELECT \\ \hline
        4 & logs (растет с 0) & 1\,500 & PK & да & INSERT \\ \hline
    \end{tabular}
    \vspace{0.5em}
\end{table}

\textbf{Пояснения:}
\begin{itemize}
    \item В графе ``TOAST'' указано, требуется ли для сценария механизм хранения крупногабаритных данных (TOAST), см.~\cite{postgres_toast}.
    \item ``PK''~--- первичный ключ; ``FK''~--- внешний ключ; ``UNIQUE''~--- уникальный индекс.
    \item Объёмы приведены в количестве строк таблицы согласно сценариям.
    \item Если поле типа JSONB или TEXT может содержать данные, превышающие одну страницу (8~КБ), PostgreSQL применяет TOAST~\cite{postgres_toast,postgres_docs}.
\end{itemize}

\subsection{Тестовая среда и ресурсы}

Тестирование проводилось на выделенном сервере, предоставленном одним из популярных провайдеров облачных решений. 
Система управления базами данных (СУБД) была развернута на операционной системе Ubuntu без дополнительных оптимизаций конфигурации, 
что позволило оценить её производительность в условиях стандартной настройки. 
Такой подход обеспечил объективность результатов тестирования, исключив влияние сторонних модификаций и специализированных тюнинговых параметров.

\insertlisting{Конфигурация тестового стенда \texttt{stand.yaml}}{code/stand.yaml}

В приведённом ниже фрагменте конфигурационного файла PostgreSQL представлены только те параметры, 
которые были явно заданы в стандартной поставке системы управления базами данных (СУБД). 
Остальные настройки соответствуют значениям по умолчанию, предусмотренным базовой конфигурацией. 
В ходе тестирования параметры СУБД не изменялись, за исключением случаев, когда их модификация была явно указана в условиях эксперимента.

\insertlisting{Часть конфигурации СУБД}{code/stand.yaml}

Более полный файл конфигурации PostgreSQL представлен в (см. \ref{app:postgresql_conf}), если в нем нет искомого параметра, 
то параметр имеет стандартное значение для данной версии СУБД.

\subsection{Сравнение результатов предсказания с реальными замерами}

\subsection{Анализ точности предсказаний}

\subsection{Анализ ошибок и отклонений}

\subsection{Обсуждение ограничений текущей реализации}

\subsection{Выводы по результатам апробации}
