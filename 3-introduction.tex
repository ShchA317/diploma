\addcontentsline{toc}{section}{\protect\numberline{}ВВЕДЕНИЕ}
\begin{center}
    \section*{\centering ВВЕДЕНИЕ}
\end{center}

В современных ин\-фор\-ма\-ци\-он\-ных си\-сте\-мах ба\-зы дан\-ных за\-ни\-ма\-ют цен\-траль\-ное ме\-сто, 
обе\-спе\-чи\-вая хра\-не\-ние, об\-ра\-бот\-ку и предос\-тав\-ле\-ние ин\-фор\-ма\-ции. При про\-ек\-ти\-ро\-ва\-нии 
и экс\-плуа\-та\-ции си\-стем управ\-ле\-ния ба\-за\-ми дан\-ных (\mbox{СУБД}) од\-ним из кри\-ти\-че\-ски 
важ\-ных фак\-то\-ров яв\-ля\-ет\-ся про\-из\-во\-ди\-тель\-ность. Про\-из\-во\-ди\-тель\-ность, в сво\-ю оче\-редь, 
во мно\-гом опре\-де\-ля\-ет\-ся эф\-фек\-тив\-но\-стью ра\-бо\-ты си\-сте\-мы вво\-да-вы\-во\-да (I/O).

Си\-сте\-ма вво\-да-вы\-во\-да от\-ве\-ча\-ет за вза\-и\-мо\-дей\-ствие меж\-ду опе\-ра\-тив\-ной па\-мя\-тью и дол\-го\-вре\-мен\-ны\-ми 
уст\-рой\-ства\-ми хра\-не\-ния дан\-ных. Нес\-мот\-ря на раз\-ви\-тие тех\-но\-ло\-гий хра\-не\-ния, та\-ких как 
твер\-до\-тель\-ные на\-ко\-пи\-те\-ли (\mbox{SSD}) и си\-сте\-мы хра\-не\-ния в опе\-ра\-тив\-ной па\-мя\-ти 
(\mbox{In-Memory Databases}), про\-бле\-ма ско\-ро\-сти и на\-дёж\-но\-сти опе\-ра\-ций чте\-ния и за\-пи\-си 
ос\-та\-ёт\-ся край\-не ак\-ту\-аль\-ной.

Осо\-бен\-но важ\-ной за\-да\-ча оцен\-ки на\-груз\-ки на под\-си\-сте\-му вво\-да-вы\-во\-да ста\-но\-вит\-ся на 
эта\-пах про\-ек\-ти\-ро\-ва\-ния но\-вых си\-стем или мас\-шта\-би\-ро\-ва\-ния су\-ще\-ству\-ющих. Не\-до\-оцен\-ка 
этой на\-груз\-ки мо\-жет при\-ве\-сти к кри\-ти\-че\-ским сбо\-ям в ра\-бо\-те при\-ло\-же\-ний, уве\-ли\-че\-нию 
вре\-ме\-ни от\-кли\-ка, по\-те\-ре дан\-ных и фи\-нан\-со\-вым убыт\-кам.

На прак\-ти\-ке оцен\-ка пред\-по\-ла\-га\-е\-мой на\-груз\-ки час\-то про\-во\-дит\-ся эм\-пи\-ри\-че\-ски либо на 
ос\-но\-ве опы\-та спе\-ци\-а\-ли\-стов, что не всег\-да поз\-во\-ля\-ет дос\-тичь не\-об\-хо\-ди\-мой точ\-но\-сти. 
Ис\-поль\-зо\-ва\-ние су\-ще\-ству\-ющих ин\-стру\-мен\-тов мо\-ни\-то\-рин\-га, та\-ких как \textit{pgbench}, 
\textit{HammerDB} или си\-сте\-мные сред\-ства ста\-ти\-сти\-ки, так\-же не ре\-ша\-ет про\-бле\-му оцен\-ки 
имен\-но пла\-ни\-ру\-е\-мой, а не фак\-ти\-че\-ской на\-груз\-ки.

Та\-ким об\-ра\-зом, су\-ще\-ству\-ет яв\-ная по\-треб\-ность в раз\-ра\-бот\-ке средств, ко\-то\-рые поз\-во\-ля\-ли 
бы про\-гно\-зи\-ро\-вать на\-груз\-ку на под\-си\-сте\-му вво\-да-вы\-во\-да Post\-gre\-SQL на ос\-но\-ве ана\-ли\-за 
струк\-ту\-ры ба\-зы дан\-ных, пред\-по\-ла\-га\-е\-мых сце\-на\-ри\-ев её ис\-поль\-зо\-ва\-ния и осо\-бен\-но\-стей 
функ\-ци\-о\-ни\-ро\-ва\-ния са\-мой \mbox{СУБД}.
\vspace{5mm}

\textbf{Цель работы} — разработка средства оценки планируемой нагрузки на систему ввода-вывода СУБД PostgreSQL.\

\vspace{5mm}

\textbf{Для достижения поставленной цели необходимо решить следующие задачи:}
\begin{itemize}[leftmargin=*,align=left]
    \item Про\-анализи\-ровать су\-ще\-ствую\-щие ме\-то\-ды оцен\-ки на\-груз\-ки на си\-сте\-му вво\-да-вы\-во\-да в кон\-тек\-сте ра\-бо\-ты \mbox{СУБД} PostgreSQL.\
    \item Вы\-я\-вить ар\-хи\-тек\-тур\-ные осо\-бен\-но\-сти Post\-gre\-SQL, вли\-яю\-щие на ха\-рак\-тер на\-груз\-ки на под\-си\-сте\-му вво\-да-вы\-во\-да.\
    \item Раз\-ра\-бо\-тать мо\-дель про\-гно\-зи\-ро\-ва\-ния пла\-ни\-ру\-е\-мой I/O-на\-груз\-ки на ос\-но\-ве ана\-ли\-за ме\-та\-дан\-ных и пред\-по\-ла\-га\-е\-мых сце\-на\-ри\-ев ра\-бо\-ты с дан\-ны\-ми.\
    \item Ре\-а\-ли\-зо\-вать про\-грамм\-ное сред\-ство, по\-зво\-ля\-ю\-щее ав\-то\-ма\-ти\-зи\-ро\-вать про\-цесс оцен\-ки.\
    \item Про\-вес\-ти тес\-ти\-ро\-ва\-ние раз\-ра\-бо\-тан\-но\-го сред\-ства на раз\-лич\-ных сце\-на\-ри\-ях ис\-поль\-зо\-ва\-ния ба\-зы дан\-ных.\
\end{itemize}

\vspace{5mm}

\textbf{Объект исследования}: процессы взаимодействия СУБД PostgreSQL с подсистемой ввода-вывода.

\textbf{Предмет исследования}: методы и средства прогнозирования планируемой нагрузки на систему ввода-вывода PostgreSQL.

\vspace{5mm}

\textbf{Актуальность темы} обусловлена необходимостью повышения надёжности проектируемых информационных систем, оптимизации их производительности, а также минимизации затрат на серверное оборудование за счёт более точного планирования ресурсов.

\vspace{5mm}

В ходе работы будут рассмотрены как существующие подходы к мониторингу и оценке нагрузки, так и предложены новые методы прогнозирования на основе анализа метаданных PostgreSQL и особенностей предполагаемой нагрузки.
